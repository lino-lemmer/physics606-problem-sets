\documentclass[11pt, ngerman, fleqn, DIV=15, headinclude]{scrartcl}

\usepackage[bibatend]{../header}

\usepackage{tikz}

\hypersetup{
    pdftitle=
}

\newcounter{totalpoints}
\newcommand\punkte[1]{#1\addtocounter{totalpoints}{#1}}

\newcounter{problemset}
\setcounter{problemset}{1}

\subject{physics606 -- Advanced Quantum Theory}
\ihead{physics606 -- Problem Set \arabic{problemset}}

\title{Problem Set \arabic{problemset}}

\publishers{Group 2 -- Swasti Belwal}
\ofoot{Group 2 -- Swasti Belwal}


\author{
    Martin Ueding \\ \small{\href{mailto:mu@martin-ueding.de}{mu@martin-ueding.de}}
    \and
    Lino Lemmer \\ \small{\href{mailto:l2@uni-bonn.de}{l2@uni-bonn.de}}
}
\ifoot{Martin Ueding, Lino Lemmer}

\ohead{\rightmark}

\begin{document}

\maketitle

\vspace{3ex}

\begin{center}
    \begin{tabular}{rrr}
        problem number & achieved points & possible points \\
        \midrule
        1 & & \punkte{8} \\
        2 & & \punkte{8} \\
        3 & & \punkte{5} \\
        4 & & \punkte{10} \\
        \midrule
        Total & & \arabic{totalpoints}
    \end{tabular}
\end{center}

\vspace{5ex}

I, Martin Ueding, would like to scan and upload the problem sets with your
corrections to my website \href{http://martin-ueding.de}{martin-ueding.de}.
There, the original problem set as well as the reviewed one will be licensed
under the “\href{http://creativecommons.org/licenses/by-sa/4.0/}{Creative
Commons Attribution-ShareAlike 4.0 International License}”. Is that okay with
you?

Yes $\Box$ \hspace{2cm} No $\Box$

\newpage

\section{Hermitean Operators}

\subsection{Real Eigenvalues}

Let the eigenvalue of the operator $Q$ on $\psi_2$ be $q_2$. For now, we assume
$q_2 \in \C$. We start with the equation~(1) from the problem set:
\begin{align*}
    \int \dif x \, \psi_1^*(x) Q \psi_2^*(x)
    &= \int \dif x \, \sbr{Q \psi_1(x)}^* \psi_2^*(x). \\
    \intertext{%
        We expand the complex conjugate in the square bracket:
    }
    \int \dif x \, \psi_1^*(x) Q \psi_2^*(x)
    &= \int \dif x \, \psi_1^*(x) Q^\dagger \psi_2^*(x). \\
    \intertext{%
        The eigenvalue of $Q^\dagger$ is the complex conjugate of the
        eigenvalue of $Q$:
    }
    q_2 \int \dif x \, \psi_1^*(x) \psi_2^*(x)
    &= q_2^* \int \dif x \, \psi_1^*(x) \psi_2^*(x) \\
    \iff q_2 &= q_2^*.
\end{align*}

This restricts the eigenvalues to $\Im(q) = 0$ which means that $q \in \R$. The
eigenvalues are all real.

\subsection{Orthogonal Eigenfunctions}

We will use the bra-ket notation here. Let $\ket n$ for $n \in \N$ (perhaps
bounded to some $N$) be the eigenstates for $Q$ with eigenvalues $q_n$. All the
$q_n$ are assume to be pairwise different.

First we have for $m, n \in \N$ which are assumed to be different:
\begin{align*}
    \braket{n | Q | m} &= q_m \braket{n | m}, \\
    \braket{m | Q | n} &= q_n \braket{m | n}.
\end{align*}

Since
\[
    \braket{n | Q | m} = \braket{m | Q | n}^*,
\]
the equations above are the complex conjugate of each other. Including the
previously proven fact that the eigenvalues are real, we can derive the
following equation:
\[
    q_m \braket{n|m} = q_n \braket{n|m}.
\]
This leaves us with:
\[
    [q_m - q_n] \braket{n|m} = 0.
\]
We assume non-degenerate eigenvalues, which means that $\braket{n|m} = 0$. The
definition of orthogonality is a vanishing scalar product, which is the case
here. Therefore $\ket n$ and $\ket m$ are orthogonal.

\section{Decomposition of a Wave Function}

\section{Angular Momentum Operator}

\section{Canonical Transformations}


\end{document}

% vim: spell spelllang=en tw=79
