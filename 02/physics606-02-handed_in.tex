\documentclass[11pt, ngerman, fleqn, DIV=15, headinclude, BCOR=1cm]{scrartcl}

\usepackage[bibatend]{../header}

\usepackage{tikz}

\usepackage[tikz]{mdframed}
\newmdtheoremenv[%
    backgroundcolor=black!5,
    innertopmargin=\topskip,
    splittopskip=\topskip,
]{theorem}{Theorem}[section]

\hypersetup{
    pdftitle=
}

\newcounter{totalpoints}
\newcommand\punkte[1]{#1\addtocounter{totalpoints}{#1}}

\newcounter{problemset}
\setcounter{problemset}{2}

\subject{physics606 -- Advanced Quantum Theory}
\ihead{physics606 -- Problem Set \arabic{problemset}}

\title{Problem Set \arabic{problemset}}

\publishers{Group 2 -- Dilege Gülmez}
\ofoot{Group 2 -- Dilege Gülmez}



\author{
    Martin Ueding \\ \small{\href{mailto:mu@martin-ueding.de}{mu@martin-ueding.de}}
    \and
    Lino Lemmer \\ \small{\href{mailto:l2@uni-bonn.de}{l2@uni-bonn.de}}
}
\ifoot{Martin Ueding, Lino Lemmer}

\ohead{\rightmark}

\begin{document}

\maketitle

\vspace{3ex}

\begin{center}
    \begin{tabular}{rrr}
        problem number & achieved points & possible points \\
        \midrule
        1 & & \punkte{7} \\
        2 & & \punkte{9} \\
        3 & & \punkte{12} \\
        \midrule
        Total & & \arabic{totalpoints}
    \end{tabular}
\end{center}

\section{Canonical Transformations and Classical Trajectories}

\subsection{Valid trajectories}

Let $(q_i(t), p_i(t))$ be a solution to the equations of motion. Then for $q_i$
we have
\begin{align*}
    \dot q_i(t) &= \cbr{q_i(t), H}. \\
    \intertext{%
        We add the same to both sides which will not make any difference:
    }
    \iff \dot q_i(t) + \epsilon \cbr{\dpd{g}{p_i}, H} &= \cbr{q_i(t), H} + \epsilon \cbr{\dpd{g}{p_i}, H}. \\
    \intertext{%
        On the left hand side we expand the Poisson bracket into its
        definition.
    }
    \iff \dot q_i(t) + \epsilon \sum_j \sbr{\dmd{g}{2}{q_j}{}{p_i}{}
        \dpd{H}{p_j}
    - \dmd{g}{2}{p_j}{}{p_i}{} \dpd{H}{q_j}} &= \cbr{q_i(t) + \epsilon \dpd{g}{p_i}, H}. \\
    \intertext{%
        The equations of motion allow us to eliminate $H$ on the left hand
        side.
    }
    \iff \dot q_i(t) + \epsilon \sum_j \sbr{\dmd{g}{2}{q_j}{}{p_i}{} \dot q_j +
    \dmd{g}{2}{p_j}{}{p_i}{} \dot p_j} &= \cbr{q_i(t) + \epsilon \dpd{g}{p_i}, H}. \\
    \intertext{%
        The term in the square bracket is
        \[
            \dod{}t \dpd{}{p_i} g\del{q_1(t), \ldots, q_n(t), p_1(t), \ldots, p_n(t)}.
        \]
        Therefore we can write this as
    }
    \iff \dot q_i(t) + \epsilon \dod{}t \dpd{g}{p_i} &= \cbr{q_i(t) + \epsilon \dpd{g}{p_i}, H}. \\
    \intertext{%
        We factor out the time derivative and yield
    }
    \iff \dod{}t \sbr{q_i(t) + \epsilon \dpd{g}{p_i}} &= \cbr{q_i(t) + \epsilon \dpd{g}{p_i}, H}. \\
    \intertext{%
        Inserting the definition of the transformation gives us
    }
    \iff \dod{}t \bar q_i(t) &= \cbr{\bar q_i(t), H}.
\end{align*}

That is the equation of motion for $\bar q_i$. The same thing can be done with
$\bar p_i$. Therefore, these infinitesimal transformations generate more valid
trajectories.

\subsection{Single particle}

It is not clear to us what $\delta$ is (scalar constant, scalar function of
something else, variational operator, …). We assume that it is a simple scalar.

The generator is $g = \delta p_k$ since
\[
    \delta = \dpd{g}{p_k}.
\]
For this to be a canonical transformation, we have to have $\{g, H\} = 0$. That
means
\[
    \{\delta p_k, H\} = \delta \dot p_k = - \dpd{H}{q_k} = 0.
\]
The hamiltonian must not depend on $q_k$ for this to work. Also since $\dot p_k
= 0$ this means that the momentum in the $k$-direction is preserved, as
Noether's theorem also says.

\section{Canonical Transformations in Quantum Mechanics}

\subsection{Transformation of the coefficients}

Equation~(3) from the problem set states:
\[
    \ket \psi = \sum_n c_n(t) \ket n,
\]
where we introduced $\ket n$ as the eigenstates to the operator $\hat g$. The
eigenvalues of $\hat g$ are $g_n$, so we simply get:
\[
    \ket{\tilde \psi}
    = \sum_n c_n(t) \hat U_g(\xi) \ket n
    = \sum_n c_n(t) g_n \ket n.
\]
Then $\tilde c_n(t) = c_n(t) g_n$ follows.

\subsection{Rotations around $z$-axis}

Let
\[
    g = \hat L_z = - \iup \hbar \dpd{}\phi
\]
be the generator now. Then the unitary operator is
\[
    \hat U_{\hat L_z}(\xi) = \exp\del{- \xi \dpd{}\phi}.
\]

We take an eigenfunction of $\hat L_z$ which is
\[
    \psi_m(\phi, t) = \exp(\iup m \phi).
\]
When we apply $\hat U_{\hat L_z}(\xi)$ to this eigenfunction, we get
\[
    \hat U_{\hat L_z}(\xi) \exp(\iup m \phi) = \exp(-\xi \iup \phi) \exp(\iup m \phi).
\]
That can be written as
\[
    \exp(-\xi \iup \phi) \exp(\iup m \phi) = \psi(\phi - \xi, t),
\]
which is a rotation around the $z$-axis with the angle $\xi$. This indeed
generates rotations.

\subsection{Rotations of superpositions}

\subsubsection{Case fixed $l$ and fixed $m$}

Our state is
\[
    \ket \psi = \ket{l, m}
\]
where $\ket{l, m}$ is an eigenstate of $\hat L^2$ and $\hat L_z$ which commutes
with the former. The part that $\hat L_z$ acts upon is $\exp(\iup m \phi)$, so
the operator $\hat U_{\hat L_z}(\xi)$ will add a phase factor of $\exp(- \iup m
\xi)$ to the whole expression. This does \emph{not} change the physical
reality, since it is just a change of the \emph{overall} phase of $\ket \psi$.

\subsubsection{Case fixed $l$ but different $m$}

Now our state is
\[
    \ket \psi = \sum_m c_m(t) \ket{l, m},
\]
where the coefficients are for the different values of $m$. The new
coefficients look like this:
\[
    \tilde c_m(t) = c_m(t) \exp(-\iup m \phi).
\]
The different summands of the superposition have a phase change which depends
on $m$. So this is no overall phase change that could be factored out. This
does change the physical reality.

\subsubsection{Case different $l$ and different $m$}

This is similar to the above case. The states are now:
\[
    \ket \psi = \sum_{l,m} c_{lm}(t) \ket{l, m},
\]
The coefficients still transform in the same $m$ dependence as before, altering
more than the overall phase of $\ket \psi$. This also changes the physical
reality.

\section{Gauge Invariance in Classical Electrodynamics}


\end{document}

% vim: spell spelllang=en tw=79
