\documentclass[11pt, ngerman, fleqn, DIV=15, headinclude, BCOR=1cm]{scrartcl}

\usepackage[bibatend]{../header}

\usepackage{tikz}

\usepackage[tikz]{mdframed}
\newmdtheoremenv[%
    backgroundcolor=black!5,
    innertopmargin=\topskip,
    splittopskip=\topskip,
]{theorem}{Theorem}[section]

\hypersetup{
    pdftitle=
}

\newcounter{totalpoints}
\newcommand\punkte[1]{#1\addtocounter{totalpoints}{#1}}

\newcounter{problemset}
\setcounter{problemset}{2}

\subject{physics606 -- Advanced Quantum Theory}
\ihead{physics606 -- Problem Set \arabic{problemset}}

\title{Problem Set \arabic{problemset}}

\publishers{Group 2 -- Dilege Gülmez}
\ofoot{Group 2 -- Dilege Gülmez}



\author{
    Martin Ueding \\ \small{\href{mailto:mu@martin-ueding.de}{mu@martin-ueding.de}}
    \and
    Lino Lemmer \\ \small{\href{mailto:l2@uni-bonn.de}{l2@uni-bonn.de}}
}
\ifoot{Martin Ueding, Lino Lemmer}

\ohead{\rightmark}

\begin{document}

\maketitle

\vspace{3ex}

\begin{center}
    \begin{tabular}{rrr}
        problem number & achieved points & possible points \\
        \midrule
        1 & & \punkte{7} \\
        2 & & \punkte{9} \\
        3 & & \punkte{12} \\
        \midrule
        Total & & \arabic{totalpoints}
    \end{tabular}
\end{center}

\section{Canonical Transformations and Classical Trajectories}

\subsection{Valid trajectories}

Let $(q_i(t), p_i(t))$ be a solution to the equations of motion. Then for $q_i$
we have
\begin{align*}
    \dot q_i(t) &= \cbr{q_i(t), H}. \\
    \intertext{%
        We add the same to both sides which will not make any difference:
    }
    \iff \dot q_i(t) + \epsilon \cbr{\dpd{g}{p_i}, H} &= \cbr{q_i(t), H} + \epsilon \cbr{\dpd{g}{p_i}, H}. \\
    \intertext{%
        On the left hand side we expand the Poisson bracket into its
        definition.
    }
    \iff \dot q_i(t) + \epsilon \sum_j \sbr{\dmd{g}{2}{q_j}{}{p_i}{}
        \dpd{H}{p_j}
    - \dmd{g}{2}{p_j}{}{p_i}{} \dpd{H}{q_j}} &= \cbr{q_i(t) + \epsilon \dpd{g}{p_i}, H}. \\
    \intertext{%
        The equations of motion allow us to eliminate $H$ on the left hand
        side.
    }
    \iff \dot q_i(t) + \epsilon \sum_j \sbr{\dmd{g}{2}{q_j}{}{p_i}{} \dot q_j +
    \dmd{g}{2}{p_j}{}{p_i}{} \dot p_j} &= \cbr{q_i(t) + \epsilon \dpd{g}{p_i}, H}. \\
    \intertext{%
        The term in the square bracket is
        \[
            \dod{}t \dpd{}{p_i} g\del{q_1(t), \ldots, q_n(t), p_1(t), \ldots, p_n(t)}.
        \]
        Therefore we can write this as
    }
    \iff \dot q_i(t) + \epsilon \dod{}t \dpd{g}{p_i} &= \cbr{q_i(t) + \epsilon \dpd{g}{p_i}, H}. \\
    \intertext{%
        We factor out the time derivative and yield
    }
    \iff \dod{}t \sbr{q_i(t) + \epsilon \dpd{g}{p_i}} &= \cbr{q_i(t) + \epsilon \dpd{g}{p_i}, H}. \\
    \intertext{%
        Inserting the definition of the transformation gives us
    }
    \iff \dod{}t \bar q_i(t) &= \cbr{\bar q_i(t), H}.
\end{align*}

That is the equation of motion for $\bar q_i$. The same thing can be done with
$\bar p_i$. Therefore, these infinitesimal transformations generate more valid
trajectories.

\subsection{Single particle}

It is not clear to us what $\delta$ is (scalar constant, scalar function of
something else, variational operator, …). We assume that it is a simple scalar.

The generator is $g = \delta p_k$ since
\[
    \delta = \dpd{g}{p_k}.
\]
For this to be a canonical transformation, we have to have $\{g, H\} = 0$. That
means
\[
    \{\delta p_k, H\} = \delta \dot p_k = - \dpd{H}{q_k} = 0.
\]
The hamiltonian must not depend on $q_k$ for this to work. Also since $\dot p_k
= 0$ this means that the momentum in the $k$-direction is preserved, as
Noether's theorem also says.

\section{Canonical Transformations in Quantum Mechanics}

\subsection{Transformation of the coefficients}

Equation~(3) from the problem set states:
\[
    \ket \psi = \sum_n c_n(t) \ket n,
\]
where we introduced $\ket n$ as the eigenstates to the operator $\hat g$. The
eigenvalues of $\hat g$ are $g_n$, so we simply get:
\[
    \ket{\tilde \psi}
    = \sum_n c_n(t) \hat U_g(\xi) \ket n
    = \sum_n c_n(t) g_n \ket n.
\]
Then $\tilde c_n(t) = c_n(t) g_n$ follows.

\subsection{Rotations around $z$-axis}

Let
\[
    g = \hat L_z = - \iup \hbar \dpd{}\phi
\]
be the generator now. Then the unitary operator is
\[
    \hat U_{\hat L_z}(\xi) = \exp\del{- \xi \dpd{}\phi}.
\]

We take an eigenfunction of $\hat L_z$ which is
\[
    \psi_m(\phi, t) = \exp(\iup m \phi).
\]
When we apply $\hat U_{\hat L_z}(\xi)$ to this eigenfunction, we get
\[
    \hat U_{\hat L_z}(\xi) \exp(\iup m \phi) = \exp(-\xi \iup \phi) \exp(\iup m \phi).
\]
That can be written as
\[
    \exp(-\xi \iup \phi) \exp(\iup m \phi) = \psi(\phi - \xi, t),
\]
which is a rotation around the $z$-axis with the angle $\xi$. This indeed
generates rotations.

\subsection{Rotations of superpositions}

\subsubsection{Case fixed $l$ and fixed $m$}

Our state is
\[
    \ket \psi = \ket{l, m}
\]
where $\ket{l, m}$ is an eigenstate of $\hat L^2$ and $\hat L_z$ which commutes
with the former. The part that $\hat L_z$ acts upon is $\exp(\iup m \phi)$, so
the operator $\hat U_{\hat L_z}(\xi)$ will add a phase factor of $\exp(- \iup m
\xi)$ to the whole expression. This does \emph{not} change the physical
reality, since it is just a change of the \emph{overall} phase of $\ket \psi$.

\subsubsection{Case fixed $l$ but different $m$}

Now our state is
\[
    \ket \psi = \sum_m c_m(t) \ket{l, m},
\]
where the coefficients are for the different values of $m$. The new
coefficients look like this:
\[
    \tilde c_m(t) = c_m(t) \exp(-\iup m \phi).
\]
The different summands of the superposition have a phase change which depends
on $m$. So this is no overall phase change that could be factored out. This
does change the physical reality.

\subsubsection{Case different $l$ and different $m$}

This is similar to the above case. The states are now:
\[
    \ket \psi = \sum_{l,m} c_{lm}(t) \ket{l, m},
\]
The coefficients still transform in the same $m$ dependence as before, altering
more than the overall phase of $\ket \psi$. This also changes the physical
reality.

\section{Gauge Invariance in Classical Electrodynamics}

\subsection{Gauge invariance}

We have, without explicit functional dependence:
\[
    \vec B = \curl \vec A
    \eqnsep
    \vec E = - \vnabla U = \dot{\vec A}
\]
And the potentials transform like so:
\[
    \vec A \to \vec A + \vnabla \lambda
    \eqnsep
    U \to U + \dot \lambda
\]

Now this is:
\[
    \vec B \to \curl \sbr{\vec A + \vnabla \lambda} = \curl \vec A + \vec 0 =
    \vec B
\]
So $\vec B$ is unchanged. Now we transform the electric field,
\[
    \vec E \to - \vnabla U + \vnabla \dot \lambda - \dot{\vec A} - \vnabla \dot
    \lambda = \vec E,
\]
which is also unchanged.

All this can be written more elegantly using differential forms. First of all
we combine $U$ and $\vec A$ into a single four-vector $\tens A$. Using the flat
isomorphism in our four dimensional Minkoswki manifold we introduce the 1-form
$A := \tens A^\flat$. Then we have the field strength tensor, a 2-form, $F :=
\dif A$. The gauge transformation is the addition of a gradient to the spacial
components and a negative time derivative of the same entity to the temporal
component. We can write this summand as the four-vector $S^i := \nabla^i
\lambda$. It depends on the signature of the metric for the components $S^\mu$
to yield the right signs to be the summands for $A^i$ and $U = A^0$. Now $S_i =
\nabla_i \lambda$ is the component way of writing a 1-form as the gradient of a
0-form (a scalar, which is what $\lambda$ is). We write this as $S = \dif
\lambda$. Now the gauge transformation is $A \to A + \dif \lambda$. Since $F =
\dif A$ and $\dif{^2} = 0$, it can be seen elegantly that $F$ is not changed in
any way:
\[
    F = \dif A \to \dif A + \dif{^2} \lambda = \dif A = F.
\]

\subsection{Homogeneous Maxwell equations}

The first equation is fulfilled since $\tens \epsilon$ is antisymmetric whereas
the $\vnabla$ commute with each other. Summation convention is implied.
\[
    \vnabla \vec B = \vnabla [\curl \vec A] = \epsilon_{ijk} \nabla_i
    \nabla_j A_k = 0.
\]

The second equation is fulfilled since a gradient has no curl and the last two
summands cancel each other.
\[
    \curl \vec E - \dot{\vec B} = - \curl \vnabla U + \curl \dot{\vec A} - \curl
    \dot{\vec A} = \vec 0 + \vec 0.
\]

Again, this can be written with differential forms, this being a simple $\dif F
= 0$. Since $F := \dif A$ it follows that $\dif{^2} A = 0$. If you write this
in components you get
\[
    \nabla_{[\alpha} F_{\beta\gamma]} = 0,
\]
where we have use the antisymmetrization brackets in the indices. For all
values for $\alpha$, $\beta$, $\gamma$ out of $\{0, 1, 2, 3\}$ you will get one
of the four equations shown in the vectorial notation.

\subsection{Lorentz gauge}

This is again pretty laborious in the SI-system and with three-vectors. First
of all, the given gauge condition is wrong. In the nice way of writing it, it
is
\[
    \partial_\mu A^\mu = 0,
\]
meaning that the four-divergence of $\tens A$ vanishes. The Coulomb gauge
similarly is $\partial_i A^i = 0$, meaning that the three-divergence of $\vec
A$ vanishes. With this notation, it is easy to see the analogy. However, this
means that in the clumsy way this is
\[
    \vnabla \vec A = - \mu_0 \varepsilon_0 \dot U,
\]
note the time derivative on the $U$!

We will start with the first inhomogeneous equation:
\begin{align*}
    \vnabla \vec E &= \frac{\varrho}{\varepsilon_0} \\
    \iff \vnabla \sbr{- \vnabla U - \dot{\vec A}} &= \frac{\varrho}{\varepsilon_0} \\
    \iff - \laplace U - \vnabla \dot{\vec A} &= \frac{\varrho}{\varepsilon_0} \\
    \intertext{%
        Now using the gradient of the correct gauge condition gives us
    }
    \iff - \laplace U - \mu_0 \varepsilon_0 \dot U &= \frac{\varrho}{\varepsilon_0}. \\
    \intertext{%
        We define
        \[
            \dalambert := \frac{1}{c^2} \dpd[2]{}t - \laplace
        \]
        in the SI-system to write this easier:
    }
    \iff \dalambert U &= \frac{\varrho}{\varepsilon_0}.
\end{align*}
This equation only depends on $\varrho \propto J^0$ and $U \propto A^0$ now,
it is decoupled from the spatial components.

Now we will tend to the second equation, the spatial part.
\begin{align*}
    \curl \vec B &= \mu_0 \vec j + \mu_0 \varepsilon_0 \dot{\vec E} \\
    \iff \curl \curl \vec A &= \mu_0 \vec j + \mu_0 \varepsilon_0
    \sbr{- \vnabla \dot U - \dot{\vec A}} \\
    \iff \curl \curl \vec A &= \mu_0 \vec j - \mu_0 \varepsilon_0 \vnabla \dot
    U - \mu_0 \varepsilon_0 \dot{\vec A} \\
    \intertext{%
        The double curl can be expanded to $\vnabla \vnabla \vec A - \laplace
        \vec A.$ This can be shown with the Levi-Civita symbol:
        \[
            [\curl\curl \vec A]_i = \epsilon_{ijk} \partial_j \epsilon_{kmn}
            \partial_m A_n = [\delta_{im} \delta_{jn} - \delta_{in}
            \delta_{jm}] \partial_j \partial_m A_n = \partial_i \partial_j A_j
            - \partial_j \partial_j A_i = [\vnabla \vnabla \vec A - \laplace
            \vec A]_i.
        \]
        We insert that into the equation now:
    }
    \iff \vnabla \vnabla \vec A - \laplace A &= \mu_0 \vec j - \mu_0 \varepsilon_0 \vnabla \dot
    U - \mu_0 \varepsilon_0 \dot{\vec A} \\
    \intertext{%
        The correct gauge condition gives us $\mu_0 \varepsilon_0 \vnabla \dot
        U - \vnabla \vnabla A$ which we use to replace any temporal-like
        quantities from the equation.
    }
    \iff \vnabla \vnabla \vec A - \laplace A &= \mu_0 \vec j + \vnabla \vnabla \vec A - \mu_0 \varepsilon_0 \dot{\vec A} \\
    \iff \mu_0 \varepsilon_0 \dot{\vec A} - \laplace A &= \mu_0 \vec j \\
    \intertext{%
        Using the definition of $\dalambert$ again here, we obtain
    }
    \iff \dalambert \vec A = \mu_0 \vec j.
\end{align*}

Setting $A^0 := cU$ and $J^0 = \varrho/c$, we can write this equation easier as
\[
    \dalambert \tens A = \mu_0 \tens J.
\]

This could even be written with differential forms:
\[
    \dif{\hodge F} = 4 \piup \hodge J,
\]
which might be the most elegant form.

\subsection{Gauge invarance of the Lagrangian}

The difference of $L$ and its transformed one is
\[
    \Deltaup L = \intop_{\R^3} \dif{^3x} \, [\varrho \dot \lambda + \vec j
    \vnabla \lambda].
\]
Please note the difference between the Laplacian $\laplace$ and the “difference
Delta” $\Deltaup$. As it currently stands, that is not the total time
derivative of another function. Therefore, this is not necessarily gauge
invariant.

The action is the time integral of the Lagrange function, so we get:
\begin{align*}
    \Deltaup S &= \intop_\R \dif t\, \Deltaup L \\
               &= \intop_{\R^4} \dif t \, \dif{^3x} \, [\varrho \dot \lambda + \vec j \vnabla
    \lambda] \\
    \intertext{%
        This will be much clearer in four dimensional notation.
    }
               &= \intop_{\R^4} \dif{^4x} \, [\varrho \dot \lambda + \vec j
    \vnabla] = \intop_{\R^4} \dif{^4x} \, J^\mu \partial_\mu \lambda \\
    \intertext{%
        The continuity equation states that $\partial_\mu J^\mu = 0$, so we can
        write
        \[
            \partial_\mu J^\mu \lambda = [\partial_\mu J^\mu] \lambda + J^\mu
            \partial_\mu \lambda = J^\mu \partial_\mu \lambda.
        \]
        We can use this to write the integral as the volume integral over a
        divergence:
    }
               &= \intop_{\R^4} \dif{^4x} \, \partial_\mu J^\mu \lambda.
    \intertext{%
        That integral can be transformed into a surface integral over $\tens J
        \lambda$. Using the assumption that the current $\tens J$ is bounded
        ($\exists R \in \R^+\colon \forall r \geq R\colon  |\tens J(r)| = 0.$)
        the whole term vanishes.
    }
    &= 0
\end{align*}

Therefore, the action is actually the same. Therefore, the physics do not
change.

\end{document}

% vim: spell spelllang=en tw=79
