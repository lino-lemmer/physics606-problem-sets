\documentclass[11pt, english, fleqn, DIV=15, headinclude, BCOR=1cm]{scrartcl}

\usepackage[
    bibatend,
    color,
]{../header}

\usepackage{tikz}

\usepackage[tikz]{mdframed}
\newmdtheoremenv[%
    backgroundcolor=black!5,
    innertopmargin=\topskip,
    splittopskip=\topskip,
]{theorem}{Theorem}[section]

\hypersetup{
    pdftitle=
}

\newcounter{totalpoints}
\newcommand\punkte[1]{#1\addtocounter{totalpoints}{#1}}

\newcounter{problemset}
\setcounter{problemset}{3}

\subject{physics606 -- Advanced Quantum Theory}
\ihead{physics606 -- Problem Set \arabic{problemset}}

\title{Problem Set \arabic{problemset}}

\publishers{Group 2 -- Dilege Gülmez}
\ofoot{Group 2 -- Dilege Gülmez}



\author{
    Martin Ueding \\ \small{\href{mailto:mu@martin-ueding.de}{mu@martin-ueding.de}}
    \and
    Lino Lemmer \\ \small{\href{mailto:l2@uni-bonn.de}{l2@uni-bonn.de}}
}
\ifoot{Martin Ueding, Lino Lemmer}

\ohead{\rightmark}

\begin{document}

\maketitle

\vspace{3ex}

\begin{center}
    \begin{tabular}{rrr}
        problem number & achieved points & possible points \\
        \midrule
        1 & & \punkte{14} \\
        2 & & \punkte{9} \\
        3 & & \punkte{10} \\
        \midrule
        Total & & \arabic{totalpoints}
    \end{tabular}
\end{center}

\section{Particle in external electromagnetic field} % GH-8

\subsection{Canonical momentum}

In general, the canonical momenta in classical mechanics are defined as
\[
    \pi_i := \frac{\partial L}{\partial \dot x^i}.
\]
We use $\vec\pi$ instead of $\vec P$ here to make the distinction to the linear
momentum $\vec p$ clearer.

Given the particular Lagrange function we have:
\[
    \pi_i = m \dot x_i + q A_i.
\]

The linear momenta are given by
\[
    p_i = m \dot x_i
\]
and differ by the $q A_i$ in the canonical momenta.

\subsection{Hamilton function}

Generally, the Hamilton function is defined as
\[
    H(q, \pi, t) := \pi_i \dot q^i(q, \pi, t) - L(q, \dot q(q, \pi, t), t).
\]
The velocities all have to be replaced with the canonical momenta. The
velocities are given by
\[
    \dot x_i = \frac 1m [\pi_i - q A_i]
\]
which need to be inserted into the Hamilton function.

The Hamilton function is
\begin{align*}
    H
    &= \pi_i \dot q^i - L \\
    &= \vec \pi \dot{\vec x} - \frac m2 \dot{\vec x}^2 + qU - q \dot{\vec x}
    \vec A \\
    &= \frac 1m \vec \pi [\vec \pi - q \vec A] - \frac1{2m} [\vec \pi - q \vec A]^2 + qU -
    \frac qm [\vec \pi - q \vec A] \vec A \\
    &= \frac 1m [\vec \pi - q \vec A] \sbr{\vec \pi - \frac12 [\vec \pi - q \vec A] -
    q \vec A} + qU \\
    &= \frac 1m [\vec \pi - q \vec A] \sbr{\vec \pi -
    q \vec A - \frac12 [\vec \pi - q \vec A]} + qU \\
    &= \frac 1{2m} [\vec \pi - q \vec A]^2 + qU.
\end{align*}

The total energy should be
\[
    \frac{1}{2m} \vec p^2 + qU,
\]
so the Hamilton function is not the total energy. Also, the rest mass would be
missing, but that is another story.

\subsection{Einstein-Lorentz-equation}

\subsubsection{Problem summary}

The Poisson bracket to evaluate is
\begin{equation}
    \label{eq:1-3-bracket}
    \dot{\vec \pi} = \{\vec \pi, H\} = \pd{\vec \pi}{x^j} \pd{H}{\pi_j} -
    \pd{\vec \pi}{\pi_j} \pd{H}{x^j}.
\end{equation}
We will do this in small parts since there are a lot of terms here. For
convenience, these are the momenta and Hamilton function again:
\[
    \pi_i = m \dot x_i + q A_i
    \eqnsep
    H = \frac 1{2m} [\vec \pi - q \vec A]^2 + qU.
\]

\subsubsection{The parts}

\paragraph{Left side}

The left side of the equation~\eqref{eq:1-3-bracket} is
\[
    \dot \pi_i = m \ddot x_i + q \dot A_i.
\]

\paragraph{First part}

The first partial derivative in equation~\eqref{eq:1-3-bracket} is
\[
    \pd{\pi_i}{x^j} = \pd{}{x^j} \sbr{m \dot x_i + q A_i(x)} = q \pd{A_i}{x^j}.
\]

\paragraph{Second part}

Now the partial derivative of the Hamilton function is needed:
\[
    \pd{H}{\pi_j}
    = \frac 1{2m} \pd{}{\pi_j} [\pi_k - q A_k][\pi^k - q A^k]
    = \frac{\pi_j}m - \frac qm A_j.
\]

\paragraph{Third part}

The next one is simply
\[
    \pd{\pi_i}{\pi_j} = \delta_i^j.
\]

\paragraph{Fourth part}

The last partial derivative requires the most work:
\begin{align*}
    \pd{H}{x^j}
    &= \frac{1}{2m} \pd{}{x^j} \sbr{ \sbr{\vec \pi - q \vec A}^2 + q U}. \\
    \intertext{%
        We expand the square, which is a scalar product, into two brackets with
        summation convention:
    }
    &= \frac{1}{2m} \pd{}{x^j} \sbr{\pi_i - q A_i} \sbr{\pi^i - q A^i} + q
    \pd{U}{x^j}. \\
    \intertext{%
        Using the product rule we can cancel the $1/2$ in front and apply the
        derivatives to the first bracket:
    }
    &= \frac{1}{m} \sbr{\pd{\pi_i}{x^j} - q \pd{A_i}{x^j}} \sbr{\pi^i - q A^i} + q
    \pd{U}{x^j}. \\
    \intertext{%
        We can now insert the derivatives of $\vec \pi$ since we have
        calculated them earlier on. We also factored out a $q$ to the front:
    }
    &= \frac{q}{m} \sbr{\pd{A_i}{x^j} - \pd{A_i}{x^j}} \sbr{\pi^i - q A^i} + q
    \pd{U}{x^j}. \\
    \intertext{%
        This way, it is easy to see that the whole expression simplifies to the
        scalar potential term:
    }
    &= q \pd{U}{x^j}.
    %&= - \frac qm \pi_i \pd{A^i}{x^j} + \frac{q^2}m A_j \pd{A^i}{x^j} +
    %q \pd{U}{x^j}
\end{align*}
This vanishing of the first part can also be made plausible when one recalls
that
\[
    \dot x_i = \frac 1m [\pi_i - q A_i].
\]
From that,
\[
    \pd{\dot x^i}{x^j} = 0
\]
can be understood faster.

\subsubsection{Combined equation}

Now that we have the parts together, we can put it into a single equation:
\begin{align*}
    m \ddot x_i - q \dot A_i &= 
    q \pd{A_i}{x^j}
    \sbr{\frac{\pi_j}m + \frac qm A_j}
    -
    \delta_i^j
    q \pd{U}{x^j}. \\
    \intertext{%
        We start by putting the $\dot A$ term to the right. Then we extract a
        factor $1/m$ out of the bracket:
    }
    \iff m \ddot x_i &= \frac qm \pd{A_i}{x^j} \sbr{\pi_j - q
    A_j} - \delta_i^j q \pd{U}{x^j} - q \dot A_i. \\
    \intertext{%
        The right side is just $\dot p_j$, the linear momentum. On the right
        side, we can insert the velocity for the bracket. We also take care of
        the $\delta$ finally:
    }
    \iff \dot p_i &= q \pd{A_i}{x^j} \dot x_j - q \pd{U}{x^i} - q \dot A_i. \\
    \intertext{%
        We factor out the $q$ and rearrange the summands:
    }
    \iff \dot p_i &= q \sbr{- \pd{U}{x^i} - \dot A_i + \pd{A_i}{x^j} \dot x_j}. \\
    \intertext{%
        We now use the second relation. The proof is at the end of this
        subsection. The equation now is
    }
    \iff \dot p_i &= q \sbr{- \pd{U}{x^i} - \sbr{\pd{A_i}t + \dot x^j
    \pd{A_i}{x^j}} + \pd{A_i}{x^j} \dot x_j} \\
    \iff \dot p_i &= q \sbr{- \pd{U}{x^i} - \pd{A_i}t}. \\
\end{align*}

% FIXME This is not correct so far.

So far, this has not worked out to be the solution given on the problem
set. I, MU, assume that I overlooked some part where I had to use product
rule or so.

\subsubsection{Proof of relations}

The first identity can be shown with index notation.
\begin{proof}
    We start in index notation:
    \begin{align*}
        \sbr{\dot{\vec x} \times [\vnabla \times \vec A]}_i
        &= \epsilon_{ijk} \dot x_j \epsilon_{kmn} \partial_m A_n. \\
        \intertext{%
            The $\tens \epsilon$ commute to
        }
        &= \epsilon_{ijk} \epsilon_{kmn} \dot x_j \partial_m A_n. \\
        \intertext{%
            We do a cyclic permutation in the indices which does not change the
            value.
        }
        &= \epsilon_{ijk} \epsilon_{mnk} \dot x_j \partial_m A_n. \\
        \intertext{%
            Now we can contract the two Levi-Civita-symbols and obtain
        }
        &= [\delta_{im} \delta_{jn} - \delta_{in} \delta_{jm}] \dot x_j
        \partial_m A_n. \\
        \intertext{%
            Applying those to the remainder gives us less indices:
        }
        &= \dot x_j \partial_i A_j - \dot x_j \partial_j A_i. \\
        \intertext{%
            In order for the derivative appearing in front of the $\dot x_j$ we
            would need to subtract the term that the product rule would
            introduce. However, that term is zero in this case since $\dot x_j$
            does not explicitly depend on $x_i$. The second summand in the
            above line can already be written in vector notation again. We
            obtain
        }
        &= \partial_i \dot x_j A_j  - \sbr{[\dot{\vec x} \vnabla] \vec A}_i. \\
        \intertext{%
            Now we can write the first summand in vectorial notation as well:
        }
        &= \sbr{\vnabla \dot{\vec x} \vec A - [\dot{\vec x} \vnabla] \vec A}_i.
    \end{align*}
    That is the right side of the identity on the problem set, just with
    slightly different notation.
\end{proof}

There is still the proof of the second relation left.
\begin{proof}
    \begin{align*}
        \dot A_i
        &= \pd{A_i}t + \pd{A_i}{x^j} \pd{x^j}{t}. \\
        \intertext{%
            Since the trajectories only depend on the time, that last
            derivative is a total one. We can therefore write it as $\dot x^j$.
        }
        &= \pd{A_i}t + \pd{A_i}{x^j} \dot x^j. \\
        \intertext{%
            We can also write this in a suggestive way like this:
        }
        &= \pd{A_i}t + \dot x^j \pd{}{x^j} A_i.
    \end{align*}
    Now one can see the whole vector identity:
    \[
        \dot{\vec A} = \pd{\vec A}{t} + [\dot{\vec x} \vnabla] \vec A.
    \]
\end{proof}

\subsection{Analogy in quantum mechanics}

In quantum mechanics, the commutator
\[
    [x^i, p_j] = \iup \hbar \delta^i_j
\]
is expected. In classical mechanics, the Poisson bracket
\[
    \{x^i, \pi_j\} = \delta^i_j
\]
should hold for the canonical coordinates.

This problem might be solved by comparing $\{\vec x, \vec p\}$ with $\{\vec x,
\vec \pi\}$. We got that $\delta^i_j$ in both cases, so we are not sure whether
this is really the right way to handle this problem.

% TODO Become sure about this.

\section{Charge conservation} % GH-9

\newcommand\cc{\mathop{}\text{c.c.}}
\newcommand\hc{\mathop{}\text{h.c.}}

\subsection{Current density}

The probability current density, which is proportional to the electric current
density (omit $q$) can be derived by introducing the continuity equation and
constructing the probability current density in a way such that it fulfils the
continuity equation. Since one has to show that said equation is fulfilled in
the next part of this problem, this is not the way to go.

From classical electrodynamics, the current density is defined as $\vec j :=
\vec v \rho$ where $\vec v$ is velocity field. Due to the uncertainty relation,
it is impossible to give an exact $\vec v(\vec x)$ which would proportional to
$\vec p(\vec x)$, since we are only dealing with non-relativistic quantum
mechanics.

So frankly, we have no idea what we are supposed to do in this part of the
problem.

% TODO Have an idea.

\subsection{Satisfaction of the continuity equation}

The Schrödinger equation says in a compact notation:
\[
    \iup \hbar \partial_0 \psi = H \psi.
\]
The complex conjugate of this gives
\[
    - \iup \hbar \partial_0 \psi^* = \psi^* H^*.
\]
From that, we can write down the partial derivatives:
\[
    \dot \psi = - \frac{\iup}{\hbar} H \psi
    \eqnsep
    \dot \psi^* = \frac{\iup}{\hbar} H^* \psi^*.
\]

We now start with the time derivative of $\varrho$ in order to match it against
the divergence of $\vec j$ later on.
\begin{align*}
    \pd{\varrho}{t}
    &= q \pd{}t \psi^* \psi \\
    \intertext{%
        The first steps needs the product rule:
    }
    &= q \sbr{\dot \psi^* \psi + \psi^* \dot \psi}. \\
    \intertext{%
        Then we can use the Schrödinger equation to give the time derivative of
        the wavefunction:
    }
    &= \frac{\iup q}{\hbar} \sbr{\psi H^* \psi^*  - \psi^* H \psi}. \\
    \intertext{%
        Factoring out a minus sign lets us write this in the notation that is
        introduced in the first part of this problem.
    }
    &= - \frac{\iup q}{\hbar} \sbr{ \psi^* H \psi - \hc} \\
    \intertext{%
        We insert the full hamiltonian:
    }
    &= - \frac{\iup q}{\hbar} \sbr{ \psi^* \sbr{\frac{1}{2m} [\vec p - q \vec
    A]^2 + q U} \psi - \hc}. \\
    \intertext{%
        The part $qU$ of the Hamiltonian is real, such that the subtraction of
        the hermitean conjugate will eliminate it. We therefore drop it and
        leave only
    }
    &= - \frac{\iup q}{\hbar} \sbr{ \psi^* \frac{1}{2m} [\vec p - q \vec
    A]^2 \psi - \hc}. \\
    &= - \frac{\iup q}{2m \hbar} \sbr{ \psi^* [\vec p^2 - q \vec p \vec A - q
    \vec A \vec p + q^2 \vec A^2] \psi - \hc}. \\
    \intertext{%
        By using the real/imaginary argument again, we can say that $\vec p^2$
        and $\vec A^2$ vanish since they are purely real.
    }
    &= \frac{\iup q^2}{2m \hbar} \sbr{ \psi^* [\vec p \vec A + \vec A \vec p]
    \psi}. \\
    \intertext{%
        We insert $\vec p = - \iup \hbar \vnabla$.
    }
    &= \frac{q^2}{2m} \sbr{ \psi^* [\vnabla \vec A + \vec A \vnabla] \psi}. \\
    \intertext{%
        The commutator of $\vec A$ and $\vnabla$ is
        \[
            [\vnabla, \vec A] = \vnabla\vec A.
        \]
        This can be seen by applying it to a wave function:
        \[
            \vnabla \vec A \psi - \vec A \vnabla \psi = [\vnabla \vec A] \psi +
            \vec A \vnabla \psi - \vec A \vnabla \psi = [\vnabla \vec A] \psi.
        \]
        Using the commutator we can simplify:
    }
    &= \frac{q^2}{2m} \sbr{\psi^* [2 \vec A \vnabla + [\vnabla \vec A]] \psi}. \\
    \intertext{%
        Now we write this in index notation, using the general relativity
        notation. There,
        \[
            Z_{,i} := \pd{Z}{x^i}.
        \]
        We end up with
    }
    &= \frac{q^2}{2m} \sbr{2 A^i \psi^* \psi_{,i} + \psi^* A^i_{,i} \psi} \\
    &= \frac{q^2}{2m} \vnabla \vec A \psi^* \psi.
\end{align*}

Now we will tend to the current density $\vec j$. The continuity equation
contains the divergence of the it, so we need to compute it. We start with the
definition:
\begin{align*}
    j^i
    &= \frac{q}{2m} \sbr{\psi^* \sbr{-\iup \hbar \partial^i - q A^i} \psi + \hc} \\
    \intertext{%
        We take the divergence. Here we have used a notation that is common in
        general relativity where there are lots of partial derivatives.
        Using that notation makes it possible to get everything into one line
        without huge brackets. Summation convention is again implied.
    }
    j^i_{,i}
    &= \frac{q}{2m} \partial_i \sbr{\psi^* \sbr{-\iup \hbar \partial^i - q A^i} \psi + \hc} \\
    \intertext{%
        Using product rule, we obtain more terms. We also but the hermitean
        conjugate to the very end outside of the brackets, the prefixed scalar
        factors are also implied.
    }
    &= \frac{q}{2m} \sbr{
        \psi^*_{,i} \sbr{-\iup \hbar \partial^i - q A^i} \psi
        + \psi^* \sbr{-\iup \hbar \laplace - q A^i_{,i}} \psi
        + \psi^* \sbr{-\iup \hbar \partial^i - q A^i} \psi_{,i}
    } + \hc \\
    \intertext{%
        The hermitean conjugate is added to this. Therefore, terms that are
        purely imaginary will be cancelled out. We do this step here to get rid
        of all the terms without $A$ in it.
    }
    &= - \frac{q^2}{2m} \sbr{
        \psi^*_{,i} A^i \psi
        + \psi^* A^i_{,i} \psi
        + \psi^* A^i \psi_{,i}
    } \\
    &= - \frac{q^2}{2m} \sbr{A^i \partial_i \psi^* \psi + A^i_{,i} \psi^* \psi}
    \\
    &= - \frac{q^2}{2m} \vnabla \vec A \psi^* \psi
\end{align*}

% XXX There are some slight glitches in the derivation. It might be good to
% work out the things the “+ h.c.” explicitly in the equations.

The sum of the derived $\dot\varrho$ and this $\vnabla \vec j$ add up to zero,
fulfilling the continuity equation.

\subsection{Gauge invariance}

$\varrho$ does not change because the added phase for $\psi$ has unit modulus.

For the current density, we have
\begin{align*}
    \vec j'
    &= \frac{q}{2m} \exp\del{- \iup \frac{q}{\hbar} \lambda} \psi^* \sbr{-\iup
    \hbar \vnabla - q \sbr{\vec A + [\vnabla \lambda]}} \exp\del{\iup
    \frac{q}{\hbar} \lambda} \psi + \hc.
    \intertext{%
        Note that the first $\vnabla$ is supposed to act on the remainder of
        the line, the $[\vnabla \lambda]$ is supposed to be self-contained. It
        is a little hard to put the scope of the differential operators into
        this when not using the component notation of general relativity.
        Either way, there will be two additional terms here. One will come from
        the product and chain rule by the first $\vnabla$. The second will come
        from the gauge field itself.
    }
    &= \vec j + \sbr{\frac{q}{2m} \psi^* \sbr{-\iup \hbar \iup \frac{q}{\hbar}
    \vnabla \lambda - q [\vnabla \lambda]}  \psi + \hc}.
    \intertext{%
        Cleaning up, it will become apparent that $\vec j$ is invariant under
        such gauge transformations.
    }
    &= \vec j +\sbr{ \frac{q}{2m} \psi^* \sbr{q [\vnabla \lambda] - q [\vnabla
    \lambda]}  \psi + \hc} \\
    &= \vec j.
\end{align*}

\section{Some Gaussian integrals} % GH-10

\subsection{First integral}

\subsubsection{The integral}

We are asked to solve
\[
    \int_0^\infty \dif x \, x \exp(-ax^2)
\]
by calculating the indefinite integral first. So we do this by using
substitution with
\[
    z := x^2
    \eqnsep
    \dif z = 2 x \, \dif x.
\]
Using that, we yield
\[
    \int \dif x \, x \exp(-ax^2) = \frac{1}{2} \int \dif z \, \exp(-az) = -
    \frac{1}{2a} \exp(-az).
\]
Evaluating this at the boundary $[0, \infty)$ we obtain the result
\[
    \frac{1}{2a}.
\]

\subsubsection{Constraints on $a$}

If the real part of $a$ is negative, the integrand is not bounded any more and
the function is not integrable any more. The result
\[
    \frac{1}{2a}
\]
would turn negative for negative $\Re a$. However, the integrand itself would
still be positive semi-definite. Therefore, this cannot be right. Unless one
goes into the realm of complex magic where
\[
    \sum_{i = 0}^\infty 2^{2i} = - \frac{1}{3}
\]
can make some sense \parencite[78]{penrose-road_to_reality}.

\subsection{2D Gaussian integral}

\begin{align*}
    |I_0(a)|^2
    &=
    \int_{_\infty}^\infty \dif x \exp(-ax^2)
    \int_{_\infty}^\infty \dif y \exp(-ay^2) \\
    &=
    \int_{_\infty}^\infty \dif x \int_{_\infty}^\infty \dif y \exp(-a[x^2+y^2]) \\
    \intertext{%
        Now $x^2 + y^2 = r^2$ in polar coordinates. The measure of the
        integration, which can be derived from Gram's determinant of the
        metric tensor, is $r \dif r \dif \phi$.
    }
    &= \int_0^\infty r \dif r \int_0^{2\piup} \dif \phi \exp(-ar^2) \\
    \intertext{%
        Since the integrand does not depend on $\phi$, we will get a simple
        scalar factor:
    }
    &= 2 \piup \int_0^\infty r \dif r \exp(-ar^2). \\
    \intertext{%
        Using the previously derived fact that this integral is $1/2a$ we can
        write down the result:
    }
    &= \frac{\piup}a.
\end{align*}
That was the square of the integral to be calculated, so the integral itself is
\[
    \sqrt{\frac\piup a}.
\]

\subsection{Third integral}

We shall compute
\begin{align*}
    I_2(a)
    &= \int_{-\infty}^\infty \dif x \, x^2 \exp(-ax^2). \\
    \intertext{%
        We can obtain a $x^2$ from the exponential function via a
        differentiation:
    }
    &= - \int_{-\infty}^\infty \dif x \od{}a \exp(-ax^2). \\
    \intertext{%
        Using the Lebeque's theorem, without checking the prerequisites, we get
    }
    &= - \od{}a \int_{-\infty}^\infty \dif x \exp(-ax^2). \\
    \intertext{%
        We already solved that integral, so we get
    }
    &= - \od{}a \sqrt{\frac{\piup}{a}}. \\
    \intertext{%
        One way of writing this would be
    }
    &= \frac 12 \sqrt{\frac{\piup}{a^3}}.
\end{align*}

\subsection{Fourth integral}

First we note the following:
\[
    ax^2 + bx = \sbr{\sqrt a x + \frac{1}{2 \sqrt a} b}^2 - \frac{b^2}{4a}.
\]

The integral to compute is
\begin{align*}
    I_0(a, b)
    &= \int_{-\infty}^\infty \dif x \exp(-ax^2 + bx). \\
    \intertext{%
        We use the completion of the square we showed above:
    }
    &= \int_{-\infty}^\infty \dif x \exp\del{-\sbr{\sqrt a x + \frac{b}{2
    \sqrt a}}^2 + \frac{b^2}{4a}}. \\
    \intertext{%
        The constant term can be put in front of the integral:
    }
    &= \exp\del{\frac{b^2}{4a}} \int_{-\infty}^\infty \dif x \exp\del{-\sbr{\sqrt a x + \frac{b}{2
    \sqrt a}}^2}. \\
    \intertext{%
        Using a substitution of the integrand with a simple shift we can remove
        the
        \[
            + \frac{b}{2 \sqrt a}
        \]
        part in the exponential function. Since it is a finite shift and the
        bounds are infinite, this will not change the value of the integral at
        all. Therefore, we use the same variable $x$ and just drop the term
        linear in $b$. We get
    }
    &= \exp\del{\frac{b^2}{4a}} \int_{-\infty}^\infty \dif x \exp(-a x^2). \\
    \intertext{%
        The integral is well known by now, we can just insert it and obtain the
        final result of
    }
    &= \exp\del{\frac{b^2}{4a}} \sqrt{\frac\piup a}.
\end{align*}

\end{document}

% vim: spell spelllang=en tw=79
