\documentclass[11pt, ngerman, fleqn, DIV=15, headinclude, BCOR=1cm]{scrartcl}

\usepackage[bibatend]{../header}

\usepackage{tikz}

\usepackage[tikz]{mdframed}
\newmdtheoremenv[%
    backgroundcolor=black!5,
    innertopmargin=\topskip,
    splittopskip=\topskip,
]{theorem}{Theorem}[section]

\hypersetup{
    pdftitle=
}

\newcounter{totalpoints}
\newcommand\punkte[1]{#1\addtocounter{totalpoints}{#1}}

\newcounter{problemset}
\setcounter{problemset}{3}

\subject{physics606 -- Advanced Quantum Theory}
\ihead{physics606 -- Problem Set \arabic{problemset}}

\title{Problem Set \arabic{problemset}}

\publishers{Group 2 -- Dilege Gülmez}
\ofoot{Group 2 -- Dilege Gülmez}



\author{
    Martin Ueding \\ \small{\href{mailto:mu@martin-ueding.de}{mu@martin-ueding.de}}
    \and
    Lino Lemmer \\ \small{\href{mailto:l2@uni-bonn.de}{l2@uni-bonn.de}}
}
\ifoot{Martin Ueding, Lino Lemmer}

\ohead{\rightmark}

\begin{document}

\maketitle

\vspace{3ex}

\begin{center}
    \begin{tabular}{rrr}
        problem number & achieved points & possible points \\
        \midrule
        1 & & \punkte{14} \\
        2 & & \punkte{9} \\
        3 & & \punkte{10} \\
        \midrule
        Total & & \arabic{totalpoints}
    \end{tabular}
\end{center}

\section{Particle in external electromagnetic field} % GH-8

\subsection{Canonical momentum}

In general, the canonical momenta in classical mechanics are defined as
\[
    \pi_i := \frac{\partial L}{\partial \dot x^i}.
\]
We use $\vec\pi$ instead of $\vec P$ here to make the distinction to the linear
momentum $\vec p$ clearer.

Given the particular Lagrange function we have:
\[
    \pi_i = m \dot x_i + q A_i.
\]

The linear momenta are given by
\[
    p_i = m \dot x_i
\]
and differ by the $q A_i$ in the canonical momenta.

\subsection{Hamilton function}

Generally, the Hamilton function is defined as
\[
    H(q, \pi, t) := \pi_i \dot q^i(q, \pi, t) - L(q, \dot q(q, \pi, t), t).
\]
The velocities all have to be replaced with the canonical momenta. The
velocities are given by
\[
    \dot x_i = \frac 1m [p_i - q A_i]
\]
which need to be inserted into the Hamilton function.

The Hamilton function is
\begin{align*}
    H
    &= \pi_i \dot q^i - L \\
    &= \vec \pi \dot{\vec x} - \frac m2 \dot{\vec x}^2 + qU - q \dot{\vec x}
    \vec A \\
    &= \frac 1m \vec \pi [\vec \pi - q \vec A] - \frac1{2m} [\vec \pi - q \vec A]^2 + qU -
    \frac qm [\vec \pi - q \vec A] \vec A \\
    &= \frac 1m [\vec \pi - q \vec A] \sbr{\vec \pi - \frac12 [\vec \pi - q \vec A] -
    q \vec A} + qU \\
    &= \frac 1m [\vec \pi - q \vec A] \sbr{\vec \pi -
    q \vec A - \frac12 [\vec \pi - q \vec A]} + qU \\
    &= \frac 1{2m} [\vec \pi - q \vec A]^2 + qU.
\end{align*}

The total energy should be
\[
    \frac{1}{2m} \vec p^2 + qU,
\]
so the Hamilton function is not the total energy. Also, the rest mass is
would be missing, but that is another story. 

\subsection{Einstein-Lorentz-equation}

The Poisson bracket to evaluate is
\[
    \dot{\vec \pi} = \{\vec \pi, H\}.
\]
We will do this in small parts since there a lot of terms here.

\begin{align*}
    \dot \pi_i
    &=
\end{align*}

\section{Charge conservation} % GH-9

\section{Some Gaussian integrals} % GH-10


\end{document}

% vim: spell spelllang=en tw=79
