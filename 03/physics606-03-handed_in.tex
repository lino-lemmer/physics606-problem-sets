\documentclass[11pt, english, fleqn, DIV=15, headinclude, BCOR=1cm]{scrartcl}

\usepackage[bibatend, color]{../header}

\usepackage{tikz}

\usepackage[tikz]{mdframed}
\newmdtheoremenv[%
    backgroundcolor=black!5,
    innertopmargin=\topskip,
    splittopskip=\topskip,
]{theorem}{Theorem}[section]

\hypersetup{
    pdftitle=
}

\newcounter{totalpoints}
\newcommand\punkte[1]{#1\addtocounter{totalpoints}{#1}}

\newcounter{problemset}
\setcounter{problemset}{3}

\subject{physics606 -- Advanced Quantum Theory}
\ihead{physics606 -- Problem Set \arabic{problemset}}

\title{Problem Set \arabic{problemset}}

\publishers{Group 2 -- Dilege Gülmez}
\ofoot{Group 2 -- Dilege Gülmez}



\author{
    Martin Ueding \\ \small{\href{mailto:mu@martin-ueding.de}{mu@martin-ueding.de}}
    \and
    Lino Lemmer \\ \small{\href{mailto:l2@uni-bonn.de}{l2@uni-bonn.de}}
}
\ifoot{Martin Ueding, Lino Lemmer}

\ohead{\rightmark}

\begin{document}

\maketitle

\vspace{3ex}

\begin{center}
    \begin{tabular}{rrr}
        problem number & achieved points & possible points \\
        \midrule
        1 & & \punkte{14} \\
        2 & & \punkte{9} \\
        3 & & \punkte{10} \\
        \midrule
        Total & & \arabic{totalpoints}
    \end{tabular}
\end{center}

\section{Particle in external electromagnetic field} % GH-8

\subsection{Canonical momentum}

In general, the canonical momenta in classical mechanics are defined as
\[
    \pi_i := \frac{\partial L}{\partial \dot x^i}.
\]
We use $\vec\pi$ instead of $\vec P$ here to make the distinction to the linear
momentum $\vec p$ clearer.

Given the particular Lagrange function we have:
\[
    \pi_i = m \dot x_i + q A_i.
\]

The linear momenta are given by
\[
    p_i = m \dot x_i
\]
and differ by the $q A_i$ in the canonical momenta.

\subsection{Hamilton function}

Generally, the Hamilton function is defined as
\[
    H(q, \pi, t) := \pi_i \dot q^i(q, \pi, t) - L(q, \dot q(q, \pi, t), t).
\]
The velocities all have to be replaced with the canonical momenta. The
velocities are given by
\[
    \dot x_i = \frac 1m [\pi_i - q A_i]
\]
which need to be inserted into the Hamilton function.

The Hamilton function is
\begin{align*}
    H
    &= \pi_i \dot q^i - L \\
    &= \vec \pi \dot{\vec x} - \frac m2 \dot{\vec x}^2 + qU - q \dot{\vec x}
    \vec A \\
    &= \frac 1m \vec \pi [\vec \pi - q \vec A] - \frac1{2m} [\vec \pi - q \vec A]^2 + qU -
    \frac qm [\vec \pi - q \vec A] \vec A \\
    &= \frac 1m [\vec \pi - q \vec A] \sbr{\vec \pi - \frac12 [\vec \pi - q \vec A] -
    q \vec A} + qU \\
    &= \frac 1m [\vec \pi - q \vec A] \sbr{\vec \pi -
    q \vec A - \frac12 [\vec \pi - q \vec A]} + qU \\
    &= \frac 1{2m} [\vec \pi - q \vec A]^2 + qU.
\end{align*}

The total energy should be
\[
    \frac{1}{2m} \vec p^2 + qU,
\]
so the Hamilton function is not the total energy. Also, the rest mass is
would be missing, but that is another story. 

\subsection{Einstein-Lorentz-equation}

The Poisson bracket to evaluate is
\begin{equation}
    \label{eq:1-3-bracket}
    \dot{\vec \pi} = \{\vec \pi, H\} = \pd{\vec \pi}{x^j} \pd{H}{\pi_j} -
    \pd{\vec \pi}{\pi_j} \pd{H}{x^j}.
\end{equation}
We will do this in small parts since there are a lot of terms here. For
convenience, these are the momenta and Hamilton function again:
\[
    \pi_i = m \dot x_i + q A_i
    \eqnsep
    H = \frac 1{2m} [\vec \pi - q \vec A]^2 + qU.
\]

The left side of the equation~\eqref{eq:1-3-bracket} is
\[
    \dot \pi_i = m \ddot x_i + q \dot A_i.
\]

The first partial derivative in equation~\eqref{eq:1-3-bracket} is
\[
    \pd{\pi_i}{x^j} = \pd{}{x^j} \sbr{m \dot x_i + q A_i(x)} = q \pd{A_i}{x^j}.
\]

Now the partial derivative of the Hamilton function is needed:
\[
    \pd{H}{\pi_j}
    = \frac 1{2m} \pd{}{\pi_j} [\pi_k - q A_k][\pi^k - q A^k]
    = \frac{\pi_j}m - \frac qm A_j.
\]

The next one is simply
\[
    \pd{\pi_i}{\pi_j} = \delta_i^j.
\]

The last partial derivative requires the most work:
\begin{align*}
    \pd{H}{x^j}
    &= \frac{1}{2m} \pd{}{x^j} \sbr{ \sbr{\vec \pi - q \vec A}^2 + q U}. \\
    \intertext{%
        We expand the square, which is a scalar product, into two brackets with
        summation convention:
    }
    &= \frac{1}{2m} \pd{}{x^j} \sbr{\pi_i - q A_i} \sbr{\pi^i - q A^i} + q
    \pd{U}{x^j}. \\
    \intertext{%
        Using the product rule we can cancel the $1/2$ in front and apply the
        derivatives to the first bracket:
    }
    &= \frac{1}{m} \sbr{\pd{\pi_i}{x^j} - q \pd{A_i}{x^j}} \sbr{\pi^i - q A^i} + q
    \pd{U}{x^j}. \\
    \intertext{%
        We can now insert the derivatives of $\vec \pi$ since we have
        calculated them earlier on. We also factored out a $q$ to the front:
    }
    &= \frac{q}{m} \sbr{\pd{A_i}{x^j} - \pd{A_i}{x^j}} \sbr{\pi^i - q A^i} + q
    \pd{U}{x^j}. \\
    \intertext{%
        This way, it is easy to see that the whole expression simplifies to the
        scalar potential term:
    }
    &= q \pd{U}{x^j}.
    %&= - \frac qm \pi_i \pd{A^i}{x^j} + \frac{q^2}m A_j \pd{A^i}{x^j} +
    %q \pd{U}{x^j}
\end{align*}
This vanishing of the first part can also be made plausible when one recalls
that
\[
    \dot x_i = \frac 1m [\pi_i - q A_i].
\]
From that,
\[
    \pd{\dot x^i}{x^j} = 0
\]
can be understood faster.

Now that we have the parts together, we can put it all together:
\begin{align*}
    m \ddot x_i - q \dot A_i &= 
    q \pd{A_i}{x^j}
    \sbr{\frac{\pi_j}m + \frac qm A_j}
    -
    \delta_i^j
    q \pd{U}{x^j}. \\
    \intertext{%
        We start by putting the $\dot A$ term to the right. Then we extract a
        factor $1/m$ out of the bracket:
    }
    \iff m \ddot x_i &= \frac qm \pd{A_i}{x^j} \sbr{\pi_j - q
    A_j} - \delta_i^j q \pd{U}{x^j} - q \dot A_i. \\
    \intertext{%
        The right side is just $\dot p_j$, the linear momentum. On the right
        side, we can insert the velocity for the bracket. We also take care of
        the $\delta$ finally:
    }
    \iff \dot p_i &= q \pd{A_i}{x^j} \dot x_j - q \pd{U}{x^i} - q \dot A_i. \\
    \intertext{%
        We factor out the $q$ and rearrange the summands:
    }
    \iff \dot p_i &= q \sbr{- \pd{U}{x^i} - \dot A_i + \pd{A_i}{x^j} \dot x_j}. \\
    \intertext{%
        We now use the second relation. The proof is at the end of this
        subsection. The equation now is
    }
    \iff \dot p_i &= q \sbr{- \pd{U}{x^i} - \sbr{\pd{A_i}t + \dot x^j
    \pd{A_i}{x^j}} + \pd{A_i}{x^j} \dot x_j} \\
    \iff \dot p_i &= q \sbr{- \pd{U}{x^i} - \pd{A_i}t}. \\
\end{align*}

\fehlt

There is still the proof of the given relation left:

\begin{proof}
    \begin{align*}
        \dot A_i
        &= \pd{A_i}t + \pd{A_i}{x^j} \pd{x^j}{t}. \\
        \intertext{%
            Since the trajectories only depend on the time, that last
            derivative is a total one. We can therefore write it as $\dot x^j$.
        }
        &= \pd{A_i}t + \pd{A_i}{x^j} \dot x^j. \\
        \intertext{%
            We can also write this in a suggestive way like this:
        }
        &= \pd{A_i}t + \dot x^j \pd{}{x^j} A_i.
    \end{align*}
    Now one can see the whole vector identity:
    \[
        \dot{\vec A} = \pd{\vec A}{t} + [\dot{\vec x} \vnabla] \vec A.
    \]
\end{proof}

\section{Charge conservation} % GH-9

\section{Some Gaussian integrals} % GH-10

\subsection{First integral}

We are asked to solve
\[
    \int_0^\infty \dif x \, x \exp(-ax^2)
\]
by calculating the indefinite integral first. So we do this by using
substitution with
\[
    z := x^2
    \eqnsep
    \dif z = 2 x \, \dif x.
\]
Using that, we yield
\[
    \int \dif x \, x \exp(-ax^2) = \frac{1}{2} \int \dif z \, \exp(-az) = -
    \frac{1}{2a} \exp(-az).
\]
Evaluating this at the boundary $[0, \infty)$ we obtain the result
\[
    \frac{1}{2a}.
\]

If the real part of $a$ is negative, the integrand is not bounded any more and
the function is not integrable any more. The result
\[
    \frac{1}{2a}
\]
would turn negative for negative $\Re a$. However, the integrand itself would
still be positive semi-definite. Therefore, this cannot be right. Unless one
goes into the realm of complex magic where
\[
    \sum_{i = 0}^\infty 2^{2i} = - \frac{1}{3}
\]
can make some sense \parencite[78]{penrose-road_to_reality}.

\subsection{2D Gaussian integral}

\begin{align*}
    |I_0(a)|^2
    &=
    \int_{_\infty}^\infty \dif x \exp(-ax^2)
    \int_{_\infty}^\infty \dif y \exp(-ay^2) \\
    &=
    \int_{_\infty}^\infty \dif x \int_{_\infty}^\infty \dif y \exp(-a[x^2+y^2]) \\
    \intertext{%
        Now $x^2 + y^2 = r^2$ in polar coordinates. The measure of the
        integration, which can be derived from Gram's determinant of the
        metric tensor, is $r \dif r \dif \phi$.
    }
    &= \int_0^\infty r \dif r \int_0^{2\piup} \dif \phi \exp(-ar^2) \\
    \intertext{%
        Since the integrand does not depend on $\phi$, we will get a simple
        scalar factor:
    }
    &= 2 \piup \int_0^\infty r \dif r \exp(-ar^2). \\
    \intertext{%
        Using the previously derived fact that this integral is $1/2a$ we can
        write down the result:
    }
    &= \frac{\piup}a.
\end{align*}
That was the square of the integral to be calculated, so the integral itself is
\[
    \sqrt{\frac\piup a}.
\]

\subsection{Third integral}

We shall compute
\begin{align*}
    I_2(a)
    &= \int_{-\infty}^\infty \dif x \, x^2 \exp(-ax^2). \\
    \intertext{%
        We can obtain a $x^2$ from the exponential function via a
        differentiation:
    }
    &= - \int_{-\infty}^\infty \dif x \od{}a \exp(-ax^2). \\
    \intertext{%
        Using the Lebeque's theorem, without checking the prerequisites, we get
    }
    &= - \od{}a \int_{-\infty}^\infty \dif x \exp(-ax^2). \\
    \intertext{%
        We already solved that integral, so we get
    }
    &= - \od{}a \sqrt{\frac{\piup}{a}}. \\
    \intertext{%
        One way of writing this would be
    }
    &= \frac 12 \sqrt{\frac{\piup}{a^3}}. \\
\end{align*}

\subsection{Fourth integral}

First we note the following:
\[
    ax^2 + bx = \sbr{\sqrt a x + \frac{1}{2 \sqrt a} b}^2 - \frac{b^2}{4a}.
\]

The integral to compute is
\begin{align*}
    I_0(a, b)
    &= \int_{-\infty}^\infty \dif x \exp(-ax^2 + bx). \\
    \intertext{%
        We use the completion of the square we showed above:
    }
    &= \int_{-\infty}^\infty \dif x \exp\del{-\sbr{\sqrt a x + \frac{b}{2
    \sqrt a}}^2 + \frac{b^2}{4a}}. \\
    \intertext{%
        The constant term can be put in front of the integral:
    }
    &= \exp\del{\frac{b^2}{4a}} \int_{-\infty}^\infty \dif x \exp\del{-\sbr{\sqrt a x + \frac{b}{2
    \sqrt a}}^2}. \\
    \intertext{%
        Using a substitution of the integrand with a simple shift we can remove
        the
        \[
            + \frac{b}{2 \sqrt a}
        \]
        part in the exponential function. Since it is a finite shift and the
        bounds are infinite, this will not change the value of the integral at
        all. Therefore, we use the same variable $x$ and just drop the term
        linear in $b$. We get
    }
    &= \exp\del{\frac{b^2}{4a}} \int_{-\infty}^\infty \dif x \exp(-a x^2). \\
    \intertext{%
        The integral is well known by now, we can just insert it and obtain the
        final result of
    }
    &= \exp\del{\frac{b^2}{4a}} \sqrt{\frac\piup a}. \\
\end{align*}

\end{document}

% vim: spell spelllang=en tw=79
