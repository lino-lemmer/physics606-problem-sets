\documentclass[11pt, english, fleqn, DIV=15, headinclude, BCOR=1.5cm]{scrartcl}

\usepackage[
    bibatend,
    color,
]{../header}

\usepackage{tikz}
\usepackage{pdflscape}

\usepackage[tikz]{mdframed}
\newmdtheoremenv[%
    backgroundcolor=black!5,
    innertopmargin=\topskip,
    splittopskip=\topskip,
]{theorem}{Theorem}[section]

\hypersetup{
    pdftitle=
}

\newcounter{totalpoints}
\newcommand\punkte[1]{#1\addtocounter{totalpoints}{#1}}

\newcounter{problemset}
\setcounter{problemset}{5}

\subject{physics606 -- Advanced Quantum Theory}
\ihead{physics606 -- Problem Set \arabic{problemset}}

\title{Problem Set \arabic{problemset}}

\publishers{Group 2 -- Dilege Gülmez}
\ofoot{Group 2 -- Dilege Gülmez}

\newmdenv[%
    backgroundcolor=black!5,
    frametitlebackgroundcolor=black!10,
    roundcorner=5pt,
    skipabove=\topskip,
    innertopmargin=\topskip,
    splittopskip=\topskip,
    frametitle={Problem statement},
    frametitlerule=true,
    nobreak=true,
]{problem}

\newmdenv[%
    backgroundcolor=white,
    frametitlebackgroundcolor=black!10,
    roundcorner=5pt,
    skipabove=\topskip,
    innertopmargin=\topskip,
    innerbottommargin=8cm,
    splittopskip=\topskip,
    frametitle={Side question},
    frametitlerule=true,
]{question}


\author{
    Martin Ueding \\ \small{\href{mailto:mu@martin-ueding.de}{mu@martin-ueding.de}}
    \and
    Lino Lemmer \\ \small{\href{mailto:l2@uni-bonn.de}{l2@uni-bonn.de}}
}
\ifoot{Martin Ueding, Lino Lemmer}

\ohead{\rightmark}

\begin{document}

\maketitle

\vspace{3ex}

\begin{center}
    \begin{tabular}{rrr}
        problem number & achieved points & possible points \\
        \midrule
        1 & & \punkte{11} \\
        2 & & \punkte{10} \\
        3 & & \punkte{13} \\
        \midrule
        total & & \arabic{totalpoints}
    \end{tabular}
\end{center}

\section{Exponentiating operators}

\begin{problem}
    The exponential of an operator is defined as
    \[
        \eup^A = \sum_{n=0}^\infty \frac{A^n}{n!}.
    \]
\end{problem}

\subsection{Inverse}

\begin{problem}
    Show that $\eup^A \eup^{-A} = 1$ using only the definition.
\end{problem}

We start with the definition
\begin{align*}
    \eup^A \eup^{-A}
    &= \sum_{n=0}^\infty \frac{A^n}{n!} \sum_{m=0}^\infty \frac{[-A]^m}{m!} \\
    \intertext{%
        and move the terms.
    }
    &= \sum_{n=0}^\infty \sum_{m=0}^\infty \frac{A^n}{n!} \frac{[-A]^m}{m!} \\
    &= \sum_{n=0}^\infty \sum_{m=0}^\infty \frac{[-1]^m A^{m+n}}{m!n!} \\
    \intertext{%
        We now think about those two sums as of a lattice where each point can
        be labelled using $(m, n)$. Currently, the summation is done in rows
        and columns. We change this to a diagonal iterations. A new parameter
        $a := m + n$ is introduced to change the summation to:
    }
    &= \sum_{a=0}^\infty \sum_{m=0}^a \frac{[-1]^m A^a}{m![a-m]!}. \\
    \intertext{%
        This is somewhat related to the argument to show that $\Q$ is still
        countable infinite. The case $a = 0$ is special and will be taken out
        of the sum, it will be just 1. We now rearrange the terms bit more and
        introduce $a!$ which cancels in total.
    }
    &= 1 + \sum_{a=1}^\infty \frac{A^a}{a!} \sum_{m=0}^a \frac{a!}{m![a-m]!}
    [-1]^m \\
    \intertext{%
        The large fraction is just the binomial coefficient. We also add a
        $1^{a-m}$.
    }
    &= 1 + \sum_{a=1}^\infty \frac{A^a}{a!} \sum_{m=0}^a \binom{a}{m} [-1]^m
    1^{a-m} \\
    \intertext{%
        Using the binomial theorem, we can write this more compact as
    }
    &= 1 + \sum_{a=1}^\infty \frac{A^a}{a!} \underbrace{[-1 + 1]^a}_{=0} \\
    \intertext{%
        which means that the second part vanishes for any $a$. Here it is clear
        that including $a = 0$ in the sum would give the indeterminate
        expression $0^0$ which would require extra treatment either way. We are
        left with
    }
    &= 1.
\end{align*}

\subsection{Sum in exponent}

\begin{problem}
    Show that
    \[
        \eup^{A+B} = \eup^A \eup^B \eup^{[A,B]}
    \]
    if the commutator $[A,B] =: c$ is a complex number.
\end{problem}

\subsection{Commutation of exponentials}

\begin{problem}
    Show that
    \[
        \eup^A \eup^B = \eup^B \eup^A \eup^{[A,B]}.
    \]
\end{problem}

In the previous problem we got
\[
    \eup^{A + B} = \eup^A \eup^B \eup^{-c/2}
\]
where $c := [A, B] \in \C$. We can just exchange $A$ and $B$ and obtain
\[
    \eup^{B + A} = \eup^B \eup^A \eup^{c/2}.
\]

Using the first equation, we can move the $\eup^{-c/2}$ to the other side and
get
\begin{align*}
    \eup^A \eup^B &= \eup^{A + B} \eup^{c/2} \\
    \intertext{%
        There, we can commute $A$ and $B$ in the \emph{sum} in the exponent.
    }
    \iff \eup^A \eup^B &= \eup^{B + A} \eup^{c/2} \\
    \intertext{%
        Using the equation we obtained from exchanging $A$ and $B$, we yield
    }
    &= \eup^B \eup^A \eup^{c/2} \eup^{c/2} \\
    &= \eup^B \eup^A \eup^{c},
\end{align*}
which is the desired result.

\subsection{Generalized commutator}

\begin{problem}
    Show by induction that
    \[
        [A, B]_n = \sum_{i=0}^n \frac{n!}{n![n-i]!} A^{n-i} B [-A]^i.
    \]
\end{problem}

The way the problem is given, a $n!$ could be canceled. We think that this
looks so close to the binomial coefficient which was already used much in the
previous problems that it probably should be $i!$ in the denominator. Using
that, we can proof the identity using induction. Therefore, we assume the
following relation:
\[
    [A, B]_n
    = \sum_{i=0}^n \frac{n!}{i![n-i]!} A^{n-i} B [-A]^i
    = \sum_{i=0}^n \binom ni A^{n-i} B [-A]^i.
\]

\begin{proof}
    We show that the relation holds for $n = 0$:
    \[
        [A, B]_0 = B
        \eqnsep
        \sum_{i=0}^0 \binom00 A^0 B [-A]^0 = B.
    \]
    This holds. Next is the induction step.
    \begin{align*}
        [A, B]_{n+1}
        &= [A, [A,B]_n] \\
        &= A [A,B]_n - [A,B]_n A \\
        \intertext{%
            We use the relation that we want to show, this is allowed in
            induction.
        }
        &= A \sum_{i=0}^n \binom{n}{i} A^{n-i} B [-A]^i - \sum_{i=0}^n
        \binom{n}{i} A^{n-i} B [-A]^i A \\
        \intertext{%
            We factor the additional factors $A$ into the expression.
        }
        &= \sum_{i=0}^n \binom{n}{i} A^{n+1-i} B [-A]^i + \sum_{i=0}^n
        \binom{n}{i} A^{n-i} B [-A]^{i+1} \\
        \intertext{%
            Now we can shift the indices a bit. We introduce
            \[
                j := i + 1
                \eqnsep
                i = j - 1
            \]
            and put that into the expression.
        }
        &= \sum_{i=0}^n \binom{n}{i} A^{n+1-i} B [-A]^i + \sum_{j=1}^{n+1}
        \binom{n}{j-1} A^{n+1-j} B [-A]^{j} \\
        \intertext{%
            We branch off the first and last summand of the first and last sum,
            \emph{respectively}. Then we combine the middle part.
        }
        &= A^{n+1} B +  \sum_{i=1}^n \sbr{\binom{n}{i} + \binom{n}{i-1}}
        A^{n+1-i} B [-A]^i + B [-A]^{n+1} \\
        \intertext{%
            There is a handy addition theorem for binomial coefficients which
            we will use for the bracket.
        }
        &= A^{n+1} B +  \sum_{i=1}^n \binom{n+1}{i} \, A^{n+1-i} B [-A]^i
        + B [-A]^{n+1} \\
        \intertext{%
            We add ones to the stray summands.
        }
        &= \binom{n+1}{0} A^{n+1} B + \sum_{i=1}^n \binom{n+1}{i} \, A^{n+1-i} B [-A]^i
        + \binom{n+1}{n+1} B [-A]^{n+1} \\
        \intertext{%
            Now we can combine it into a single sum.
        }
        &= \sum_{i=0}^{n+1} \binom{n+1}{i} \, A^{n+1-i} B [-A]^i \\
    \end{align*}
    That is exactly what has to be shown. Therefore, this relation holds for
    any $n$.
\end{proof}

Now that we got this identity at our hands, we can aim for the actual task at
hand:

\begin{problem}
    Show that
    \[
        \eup^A B \eup^{-A} = \sum_{n = 0}^\infty \frac{1}{n!} [A, B]_n.
    \]
\end{problem}

We start with the left side of this equation:
\begin{align*}
    \eup^A B \eup^{-A}
    &= \sum_{d=0}^\infty \frac{A^d}{d!} B \sum_{f=0}^\infty \frac{[-A]^f}{f!}.
    \\
    \intertext{%
        We interchange the summations again and group the terms.
    }
    &= \sum_{d=0}^\infty \sum_{f=0}^\infty \frac{1}{f!d!} A^d B [-A]^f \\
    \intertext{%
        By now, the change in the lattice traversal was applied many times. We
        introduce
        \[
            i := f
            \eqnsep
            n := d + f
        \]
    }
    &= \sum_{n=0}^\infty \sum_{i=0}^n \frac{1}{f!d!} A^d B [-A]^f \\
\end{align*}

\section{First oder time dependent perturbation theory}

\section{Perturbed harmonic oscillator}


\end{document}

% vim: spell spelllang=en tw=79
