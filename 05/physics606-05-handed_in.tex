\documentclass[11pt, english, fleqn, DIV=15, headinclude, BCOR=1.5cm]{scrartcl}

\usepackage[
    bibatend,
    color,
]{../header}

\usepackage{tikz}
\usepackage{pdflscape}

\usepackage[tikz]{mdframed}
\newmdtheoremenv[%
    backgroundcolor=black!5,
    innertopmargin=\topskip,
    splittopskip=\topskip,
]{theorem}{Theorem}[section]

\hypersetup{
    pdftitle=
}

\newcounter{totalpoints}
\newcommand\punkte[1]{#1\addtocounter{totalpoints}{#1}}

\newcounter{problemset}
\setcounter{problemset}{5}

\subject{physics606 -- Advanced Quantum Theory}
\ihead{physics606 -- Problem Set \arabic{problemset}}

\title{Problem Set \arabic{problemset}}

\publishers{Group 2 -- Dilege Gülmez}
\ofoot{Group 2 -- Dilege Gülmez}

\newmdenv[%
    backgroundcolor=black!5,
    frametitlebackgroundcolor=black!10,
    roundcorner=5pt,
    skipabove=\topskip,
    innertopmargin=\topskip,
    splittopskip=\topskip,
    frametitle={Problem statement},
    frametitlerule=true,
]{problem}

\newmdenv[%
    backgroundcolor=white,
    frametitlebackgroundcolor=black!10,
    roundcorner=5pt,
    skipabove=\topskip,
    innertopmargin=\topskip,
    innerbottommargin=8cm,
    splittopskip=\topskip,
    frametitle={Side question},
    frametitlerule=true,
    nobreak=true,
]{question}


\author{
    Martin Ueding \\ \small{\href{mailto:mu@martin-ueding.de}{mu@martin-ueding.de}}
    \and
    Lino Lemmer \\ \small{\href{mailto:l2@uni-bonn.de}{l2@uni-bonn.de}}
}
\ifoot{Martin Ueding, Lino Lemmer}

\ohead{\rightmark}

\begin{document}

\maketitle

\vspace{3ex}

\begin{center}
    \begin{tabular}{rrr}
        problem number & achieved points & possible points \\
        \midrule
        1 & & \punkte{11} \\
        2 & & \punkte{10} \\
        3 & & \punkte{13} \\
        \midrule
        total & & \arabic{totalpoints}
    \end{tabular}
\end{center}

\section{Exponentiating operators}

\begin{problem}
    The exponential of an operator is defined as
    \[
        \eup^A = \sum_{n=0}^\infty \frac{A^n}{n!}.
    \]
\end{problem}

\subsection{Inverse}

\begin{problem}
    Show that $\eup^A \eup^{-A} = 1$ using only the definition.
\end{problem}

We start with the definition
\begin{align*}
    \eup^A \eup^{-A}
    &= \sum_{n=0}^\infty \frac{A^n}{n!} \sum_{m=0}^\infty \frac{[-A]^m}{m!} \\
    \intertext{%
        and move the terms.
    }
    &= \sum_{n=0}^\infty \sum_{m=0}^\infty \frac{A^n}{n!} \frac{[-A]^m}{m!} \\
    &= \sum_{n=0}^\infty \sum_{m=0}^\infty \frac{[-1]^m A^{m+n}}{m!n!} \\
    \intertext{%
        We now think about those two sums as of a lattice where each point can
        be labelled using $(m, n)$. Currently, the summation is done in rows
        and columns. We change this to a diagonal iterations. A new parameter
        $a := m + n$ is introduced to change the summation to:
    }
    &= \sum_{a=0}^\infty \sum_{m=0}^a \frac{[-1]^m A^a}{m![a-m]!}. \\
    \intertext{%
        This is somewhat related to the argument to show that $\Q$ is still
        countable infinite. The case $a = 0$ is special and will be taken out
        of the sum, it will be just 1. We now rearrange the terms bit more and
        introduce $a!$ which cancels in total.
    }
    &= 1 + \sum_{a=1}^\infty \frac{A^a}{a!} \sum_{m=0}^a \frac{a!}{m![a-m]!}
    [-1]^m \\
    \intertext{%
        The large fraction is just the binomial coefficient. We also add a
        $1^{a-m}$.
    }
    &= 1 + \sum_{a=1}^\infty \frac{A^a}{a!} \sum_{m=0}^a \binom{a}{m} [-1]^m
    1^{a-m} \\
    \intertext{%
        Using the binomial theorem, we can write this more compact as
    }
    &= 1 + \sum_{a=1}^\infty \frac{A^a}{a!} \underbrace{[-1 + 1]^a}_{=0} \\
    \intertext{%
        which means that the second part vanishes for any $a$. Here it is clear
        that including $a = 0$ in the sum would give the indeterminate
        expression $0^0$ which would require extra treatment either way. We are
        left with
    }
    &= 1.
\end{align*}

\section{First oder time dependent perturbation theory}

\section{Perturbed harmonic oscillator}


\end{document}

% vim: spell spelllang=en tw=79
