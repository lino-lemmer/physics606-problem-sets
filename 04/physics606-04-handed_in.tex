\documentclass[11pt, english, fleqn, DIV=15, headinclude, BCOR=1.5cm]{scrartcl}

\usepackage[
    bibatend,
    color,
]{../header}

\usepackage{tikz}
\usepackage{pdflscape}

\usepackage[tikz]{mdframed}
\newmdtheoremenv[%
    backgroundcolor=black!5,
    innertopmargin=\topskip,
    splittopskip=\topskip,
]{theorem}{Theorem}[section]

\hypersetup{
    pdftitle=
}

\newcounter{totalpoints}
\newcommand\punkte[1]{#1\addtocounter{totalpoints}{#1}}

\newcounter{problemset}
\setcounter{problemset}{4}

\subject{physics606 -- Advanced Quantum Theory}
\ihead{physics606 -- Problem Set \arabic{problemset}}

\title{Problem Set \arabic{problemset}}

\publishers{Group 2 -- Dilege Gülmez}
\ofoot{Group 2 -- Dilege Gülmez}

\newmdenv[%
    backgroundcolor=black!5,
    frametitlebackgroundcolor=black!10,
    roundcorner=5pt,
    skipabove=\topskip,
    innertopmargin=\topskip,
    splittopskip=\topskip,
    frametitle={Problem statement},
    frametitlerule=true,
]{problem}

\newmdenv[%
    backgroundcolor=white,
    frametitlebackgroundcolor=black!10,
    roundcorner=5pt,
    skipabove=\topskip,
    innertopmargin=\topskip,
    innerbottommargin=8cm,
    splittopskip=\topskip,
    frametitle={Side question},
    frametitlerule=true,
    nobreak=true,
]{question}


\author{
    Martin Ueding \\ \small{\href{mailto:mu@martin-ueding.de}{mu@martin-ueding.de}}
    \and
    Lino Lemmer \\ \small{\href{mailto:l2@uni-bonn.de}{l2@uni-bonn.de}}
}
\ifoot{Martin Ueding, Lino Lemmer}

\ohead{\rightmark}

\begin{document}

\maketitle

\vspace{3ex}

\begin{center}
    \begin{tabular}{rrr}
        problem number & achieved points & possible points \\
        \midrule
        1 & & \punkte{9} \\
        2 & & \punkte{5} \\
        3 & & \punkte{8} \\
        \midrule
        Total & & \arabic{totalpoints}
    \end{tabular}
\end{center}

In order to make this document a bit more self-sufficient, we summarize the
problem statements in front of our solutions. That will hopefully allow for
better presentation of the results.

\section{Propagator of the harmonic oscillator} % GH-11

\begin{problem}
    Let $L$ be the Lagrangian of the harmonic oscillator in one dimension:
    \[
        L = \frac 12 m [\dot x^2 - \omega^2 x^2].
    \]
\end{problem}

\subsection{Classical action}

\begin{problem}
    Show that the classical action $S_\text{cl}(x, x', t)$ has the given form.
\end{problem}

The action is defined as:
\[
    S = \int_0^t \dif t' \, L\del{x(t'), \dot x(t'), t'}.
\]

We use the hint that is given on the problem set and express the trajectory as
a linear combination of basis trajectories:
\[
    x(t') = a \cos(\omega t') + b \sin(\omega t').
\]
The boundary conditions are a little strange. In the wording, it says that the
particle goes from point $x_0$ at $t = 0$ to the point $x$ at time $t$. Then
the action is written with dependency on $x$ and $x'$. We will use $x(0) =:
x_0$ and $x(t) =: x_1$ here. This already is the boundary condition, so $a$ and
$b$ are:
\[
    a = x_0
    \eqnsep
    b = \frac{x_1 - x_0 \cos(\omega t)}{\sin(\omega t)}.
\]

This can now be inserted into the Lagrangian function:
\begin{align*}
    S_\text{cl}(x_0, x_1, t)
    &= \int_0^t \dif t' \, L\del{x(t'), \dot x(t'), t'} \\
    &= \frac{m}{2} \int_0^t \dif t' \,
    \sbr{
        \sbr{- a \omega \sin(\omega t') + b \omega \cos(\omega t')}^2
        - \omega^2 \sbr{a \cos(\omega t') + b \sin(\omega t')}^2
    } \\
    \intertext{%
        We expand all the squares and regroup the terms again. Currently, all
        arguments for sine and cosine are $\omega t'$, so we will omit them for
        a more compact view of the expression. When the arguments change in the
        next steps, we will show them explicitly. Functional arguments will
        always be denoted with round parentheses and never with square
        brackets.
    }
    &= \frac{m}{2} \int_0^t \dif t' \,
    \sbr{
        [a^2 - b^2][\sin^2 - \cos^2] - 4 a b \cos \sin
    } \\
    \intertext{%
        Now we perform the integration. The indefinite integral to $\sin^2 -
        \cos^2$ is given by
        \[
            - \frac{\sin(2\omega t')}{2\omega} = - \frac{2 \cos \sin}{2\omega}.
        \]
        The integral of $\cos\sin$ is given by
        \[
            - \frac{\cos^2}{2\omega}.
        \]
        The use those and insert into the previous equation.
    }
    &= - \frac{m \omega^2}{2}
    \sbr{
        [a^2 - b^2]\frac{\cos\sin}{\omega} - 2 a b \frac{\cos^2}{\omega}
    }_0^t \\
    \intertext{%
        Now we evaluate at $t' = t$ and $t' = 0$. Since all the arguments
        change from $\omega t'$ to $\omega t$, we redefine the omission and
        still omit all the arguments.
    }
    &= - \frac{m \omega^2}{2}
    \sbr{
        [a^2 - b^2]\frac{\cos\sin}{\omega} - 2 a b \frac{\cos^2 - 1}{\omega}
    } \\
    \intertext{%
        Now we insert $a$ and $b$ explicitly.
    }
    &= - \frac{m \omega^2}{2}
    \sbr{
        \sbr{x_0^2 - \sbr{\frac{x_1 - x_0 \cos}{\sin}}^2}\frac{\cos\sin}{\omega} -
        2 x_0 \frac{x_1 - x_0 \cos}{\sin} \frac{\cos^2 - 1}{\omega}
    } \\
    \intertext{%
        We get rid of the minus in front and cancel one $\omega$.
    }
    &= \frac{m \omega}{2}
    \sbr{
        \sbr{- x_0^2 + \sbr{\frac{x_1 - x_0 \cos}{\sin}}^2}\cos\sin +
        2 x_0 \frac{x_1 - x_0 \cos}{\sin} [\cos^2 - 1]
    } \\
    \intertext{%
        The sine in the first denominator is squared out. Then the first
        summand will be brought up to a fraction with the denominator $\sin^2$.
        One sine is pulled out front, the other cancels the sine behind the
        bracket. The sine in the second term is pulled out as well.
    }
    &= \frac{m \omega}{2 \sin}
    \sbr{
        \sbr{- x_0^2 \sin^2 + [x_1 - x_0 \cos]^2}\cos +
        2 x_0 [x_1 - x_0 \cos][\cos^2 - 1]
    } \\
    \intertext{%
        We swap summands, and replace the very last bracket with a sine
        squared.
    }
    &= \frac{m \omega}{2 \sin}
    \sbr{
        \sbr{[x_1 - x_0 \cos]^2 - x_0^2 \sin^2}\cos +
        2 x_0 [x_1 - x_0 \cos]\sin^2
    } \\
    &= \frac{m \omega}{2 \sin}
    \sbr{
        \sbr{x_1^2 - 2 x_0 x_1 \cos + x_0^2 \cos^2 - x_0^2 \sin^2}\cos +
        [2 x_0 x_1 - 2 x_0^2 \cos] \sin^2
    } \\
    &= \frac{m \omega}{2 \sin}
    \sbr{
        x_1^2 \cos - 2 x_0 x_1 \cos^2 + x_0^2 \cos^3 - x_0^2 \cos \sin^2 +
        2 x_0 x_1 \sin^2 - 2 x_0^2 \cos \sin^2
    } \\
    &= \frac{m \omega}{2 \sin}
    \sbr{
        x_0^2 [\cos^3 - 2 \cos \sin^2 - \cos \sin^2]
        + x_1^2 \cos
        + 2 x_0 x_1 [\sin^2 - \cos^2]
    } \\
    &= \frac{m \omega}{2 \sin}
    \sbr{
        x_0^2 \cos [\cos^2 - 3 \sin^2]
        + x_1^2 \cos
        + 2 x_0 x_1 [\sin^2 - \cos^2]
    }
\end{align*}

This does not quite work out.

\subsection{Separation}

\begin{problem}
    Show that the quantum mechanical propagator can be split up into a product
    of a time-only factor and the classical action:
    \[
        U(x_0, x_1, t) := \int_{x_0}^{x_1} \mathcal D x \,
        \exp\del{\frac\iup\hbar \int_0^t \dif t' \, L\del{x(t'), \dot x(t')}} =
        F(t) \exp\del{\frac\iup\hbar S_\text{cl}}.
    \]
\end{problem}

We begin with the definition:
\begin{align*}
    U(x_0, x_1, t) 
    &= \int_{x_0}^{x_1} \mathcal D x \,
    \exp\del{\frac\iup\hbar \int_0^t \dif t' \, L\del{x(t'), \dot x(t')}}.
    \intertext{%
        It is suggested to write $x(t') = x_\text{sc}(t') + y(t)$ where $y(0) =
        y(t) = 0$ is fulfilled.
    }
    &= \int_0^0 \mathcal D y \,
    \exp\del{\frac\iup\hbar \int_0^t \dif t' \, L\del{x_\text{sc}(t') + y(t'), \dot
    x_\text{sc}(t') + \dot y(t')}}. \\
    \intertext{%
        This looks a bit like the variational calculus approach that one can
        take to derive the Euler Lagrange equations. To obtain terms that are
        linear in $y$ and $\dot y$, we do an expansion of $L$ around $y \equiv
        0$. All instances of $L$ are meant to be evaluated like
        \[
            L\del{x_\text{sc}(t'), \dot x_\text{sc}(t')}.
        \]
        This gives:
    }
    &= \int_0^0 \mathcal D y \,
    \exp\del{\frac\iup\hbar \int_0^t \dif t' \, \sbr{
        L
        + \pd Lx y + \pd L{\dot x} \dot y
        + \mathcal O(y^2) + \mathcal O(\dot y^2)
    }}. \\
    \intertext{%
        Since the first part, the $L$, does not depend on the non-classical
        component of $x$ any more, we can pull in front of the $\mathcal D y$
        integral. The time integration of the Lagrangian over the classical
        path just gives the classical action.
    }
    &= \exp\del{\frac\iup\hbar S_\text{cl}} \int_0^0 \mathcal D y \,
    \exp\del{\frac\iup\hbar \int_0^t \dif t' \, \sbr{
        \pd Lx y + \pd L{\dot x} \dot y
        + \mathcal O(y^2) + \mathcal O(\dot y^2)
    }}. \\
    \intertext{%
        The term linear in $\dot y$ will now be partially integrated. The total
        time derivative is contained in the bracket and does not act on the $y$
        outside of the bracket. Looking at the interesting part, we have
        \[
            \int_0^t \dif t' \, \sbr{
                \pd Lx y + \pd L{\dot x} \dot y
            }
            =
            \int_0^t \dif t' \, \sbr{
                \sbr{\pd Lx - \od{}{t'} \pd L{\dot x}} y
            } + \sbr{\pd{L}{\dot x} y}_0^t.
        \]
        The term 
        \[
            \sbr{\pd Lx - \od{}{t'} \pd L{\dot x}} y
        \]
        is zero since the $L$ are evaluated at the classical trajectory only.
        There, the Euler Lagrange equations hold. The surface term
        \[
            \sbr{\pd{L}{\dot x} y}_0^t
        \]
        is also zero because of the definition of $y$. The remainder of the
        integral therefore is:
    }
    &= \exp\del{\frac\iup\hbar S_\text{cl}} \int_0^0 \mathcal D y \,
    \exp\del{\frac\iup\hbar \int_0^t \dif t' \, \sbr{
        \mathcal O(y^2) + \mathcal O(\dot y^2)
    }}. \\
    \intertext{%
        The remaining integration that is denoted with $\int \mathcal D y$ does
        not depend on the starting and ending points $x_0$ and $x_1$ since it
        is only concerned with the finite variation $y$. The only free variable
        in there is the $t$, so we can write this whole thing, whatever it
        actually is, as an ominous $F(t)$:
    }
    &= \exp\del{\frac\iup\hbar S_\text{cl}} F(t).
\end{align*}

\section{Ordinary path integral from phase space path integral} % GH-12

\begin{problem}
    Given the relation
    \[
        \lim_{N \to \infty} \prod_{k=1}^{N-1} \int \dif x_k \prod_{k=1}^N \int
        \frac{\dif p_k}{2\piup \hbar} \exp\del{
            -\frac\iup\hbar \sum_{k = 1}^N \sbr{
                \frac{\epsilon p_k^2}{2m} - p_k [x_k - x_{k - 1}] + \epsilon
        V(x_{k-1})}}.
    \]
    Show that this reduces to
    \[
        \frac 1B \lim_{N \to \infty} \prod_{k=1}^{N-1} \int \frac{\dif x_k}{B} 
        \exp\del{
            \frac\iup\hbar \sum_{k = 1}^N \sbr{
                \frac{m}{2} \frac{[x_k - x_{k - 1}]^2}\epsilon - \epsilon
        V(x_{k-1})}}
    \]
    where
    \[
        B = \sqrt{\frac{2\piup \hbar \iup}{m}}.
    \]
\end{problem}

\begin{landscape}
It was not quite clear to us, how the multiple usages of the index $k$ are
meant. The interpretation that got us to the right result is written in our
notation as
\begin{align*}
    U(x, x', t)
    &= \lim_{N \to \infty} \sbr{\prod_{k=1}^{N-1} \int \dif x_k}
    \sbr{\prod_{k=1}^N \int \frac{\dif p_k}{2\piup \hbar}} \exp\del{
    -\frac\iup\hbar \sum_{k = 1}^N \sbr{ \frac{\epsilon p_k^2}{2m} - p_k [x_k -
    x_{k - 1}] + \epsilon V(x_{k-1})}}. \\
    \intertext{%
        That way, the $k$ of each summation or product sign are contained
        within a scope. The sum in the exponential function can be expanded
        into a product of exponential functions. The resulting product sign
        would be the same as the one for the $\dif p_k$, so we join those
        together. The second product sign is meant to act on all the factors
        behind it, i.e.\ to the next plus or minus sign. We yield
    }
    &= \lim_{N \to \infty} \sbr{\prod_{k=1}^{N-1} \int \dif x_k}
    \prod_{k=1}^N \int \frac{\dif p_k}{2\piup \hbar}
    \exp\del{
    -\frac\iup\hbar \sbr{ \frac{\epsilon p_k^2}{2m} - p_k [x_k -
    x_{k - 1}] + \epsilon V(x_{k-1})}}
    . \\
    \intertext{%
        We rearrange the terms in the exponential function a bit to give the
        same signature the Gaussian integrals on problem set~3 have. Since the
        potential $V$ term does not depend on any of the momenta, we can safely
        pull this in front of the integral. Now we have
    }
    &= \lim_{N \to \infty} \sbr{\prod_{k=1}^{N-1} \int \dif x_k}
    \prod_{k=1}^N
    \exp\del{ -\frac{\iup\epsilon}\hbar V(x_{k-1})}
    \int \frac{\dif p_k}{2\piup \hbar}
    \exp\del{
        -\frac{\iup\epsilon}{2m\hbar} p_k^2 + \frac{\iup}{\hbar} [x_k -
    x_{k - 1}] p_k }
    , \\
    \intertext{%
        where we identify
        \[
            a := \frac{\iup\epsilon}{2m\hbar}
            \eqnsep
            b := \frac{\iup}{\hbar} [x_k - x_{k - 1}].
        \]
        Using the results from problem set~3, we get
    }
    &= \lim_{N \to \infty} \sbr{\prod_{k=1}^{N-1} \int \dif x_k}
    \prod_{k=1}^N
    \sqrt{\frac{m}{2 \piup \iup \epsilon \hbar}}
    \exp\del{ -\frac{\iup\epsilon}\hbar V(x_{k-1})}
    \exp\del{ \frac{\iup m}{2\epsilon \hbar} [x_k - x_{k - 1}]^2 }
    . \\
    \intertext{%
        We pull the square root in front of the product sign, adding the power
        $N$. Then we collapse the two exponential functions into a single one.
        To match the form given on the problem set we factor out $\iup/\hbar$.
        Then we incorporate the product sign into the exponential function in
        the form of a summation sign. With these modifications, we obtain
    }
    &= \lim_{N \to \infty} \sbr{\prod_{k=1}^{N-1} \int \dif x_k}
    \sqrt{\frac{m}{2 \piup \iup \epsilon \hbar}}^N
    \exp\del{
        \frac{\iup}{\hbar}
        \sum_{k=1}^N
        \del{
            \frac{m}{2} \frac{[x_k - x_{k - 1}]^2}\epsilon
            - \epsilon V(x_{k-1})
        }
    }
    . \\
    \intertext{%
        The product sign that is left only goes up to $N-1$, so that it only
        contains the power $N-1$. Therefore, we need to split up the square
        root into two parts.
    }
    &= \lim_{N \to \infty} \sqrt{\frac{m}{2 \piup \iup \epsilon \hbar}}
    \sbr{\prod_{k=1}^{N-1} \sqrt{\frac{m}{2 \piup \iup \epsilon \hbar}} \int \dif x_k}
        \exp\del{
        \frac{\iup}{\hbar}
        \sum_{k=1}^N
        \sbr{
            \frac{m}{2} \frac{[x_k - x_{k - 1}]^2}\epsilon
            - \epsilon V(x_{k-1})
        }
    }
    . \\
    \intertext{%
        The definition of $B$ on the problem set does not have the $\epsilon$
        we have. We cannot pull it in front of the limit since $\epsilon$
        depends on $N$. Other than that, we got the result from the problem
        set.
    }
    &= \lim_{N \to \infty} \frac{1}{B \sqrt\epsilon}
    \sbr{\prod_{k=1}^{N-1} \int \frac{\dif x_k}{B \sqrt\epsilon}}
        \exp\del{
        \frac{\iup}{\hbar}
        \sum_{k=1}^N
        \sbr{
            \frac{m}{2} \frac{[x_k - x_{k - 1}]^2}\epsilon
            - \epsilon V(x_{k-1})
        }
    }
    .
\end{align*}
\end{landscape}

\section{Path integral with vector potential} % GH-13

\begin{problem}
    Show that
    \[
        \psi(x, \epsilon)
        = \int_\R \dif \eta \, \psi(x + \eta, 0) \exp\del{
            \frac\iup\hbar \sbr{\frac{m \eta^2}{2\epsilon} + q \epsilon
            \frac\eta\epsilon A(x + \alpha \eta, 0)}
        }
    \]
    is equivalent to
    \[
        \psi(x, \epsilon) = \psi(x, 0) - \iup \frac\epsilon\hbar \hat H \psi(x, 0).
    \]
\end{problem}

The expansions are:
\begin{gather*}
    \psi(x + \eta, 0) = \psi(x, 0) + \psi'(x, 0) \eta + \frac 12 \psi''(x, 0)
    \eta^2 + \mathcal O(\eta^3) \\
    A(x + \eta, 0) = A(x, 0) + A'(x, 0) \alpha \eta + \frac 12 A''(x, 0)
    [\alpha\eta]^2 + \mathcal O(\eta^3)
\end{gather*}

We start with the given expression,
\begin{align*}
    \psi(x, \epsilon)
    &= \int_\R \dif \eta \, \psi(x + \eta, 0) \exp\del{
        \frac\iup\hbar \sbr{\frac{m \eta^2}{2\epsilon} + q \epsilon
        \frac\eta\epsilon A(x + \alpha \eta, 0)}
    }. \\
    \intertext{%
        We insert the expansion of $\psi$ and $A$.
    }
    &\approx \int_\R \dif \eta \, \sbr{\psi(x, 0) + \psi'(x, 0) \eta + \frac 12 \psi''(x, 0) \eta^2} \\
    &\quad\cdot\exp\del{ \frac\iup\hbar \sbr{\frac{m \eta^2}{2\epsilon} + q \epsilon \frac\eta\epsilon \sbr{A(x, 0) + A'(x, 0) \alpha \eta + \frac 12 A''(x, 0) [\alpha\eta]^2}} } \\
    \intertext{%
        Since all the function arguments are $(x, 0)$, we drop them.
    }
    &= \int_\R \dif \eta \, \sbr{\psi + \psi' \eta + \frac 12 \psi'' \eta^2} \exp\del{ \frac\iup\hbar \sbr{\frac{m \eta^2}{2\epsilon} + q \epsilon \frac\eta\epsilon \sbr{A + A' \alpha \eta + \frac 12 A'' [\alpha\eta]^2}} }
    \intertext{%
        We rearrange the terms in the exponential function. While we do that,
        we drop the $\eta^3$ term that would come with $A''$. Then we bring it
        into the $\exp(-ax^2 + bx)$ form to use the solution formulas later on.
    }
    &= \int_\R \dif \eta \, \sbr{\psi + \psi' \eta + \frac 12 \psi'' \eta^2}
    \exp\del{
        - \sbr{ - \frac{\iup m}{2 \epsilon \hbar} - \frac{\iup q \alpha A'}{\hbar}}
        \eta^2 + \frac{\iup q A}{\hbar} \eta
    } \\
    \intertext{%
        We now expand the first bracket and use the linearity of the integral
        to get three of them.
    }
    &= \psi \int_\R \dif \eta \, \exp(\ldots)
    + \psi' \int_\R \dif \eta \, \eta \exp(\ldots)
    + \frac 12 \psi'' \int_\R \dif \eta \, \eta^2 \exp(\ldots)
    \intertext{%
        We now explicitly define $a$ and $b$ to be
        \[
            a := - \frac{\iup m}{2 \epsilon \hbar} - \frac{\iup q \alpha A'}{\hbar}
            \eqnsep
            b := \frac{\iup q A}{\hbar}.
        \]
        Then we can use the different solutions formulas for the various
        integrals:
    }
    &= \psi \exp\del{\frac{b^2}{4a}} \sqrt{\frac{\piup}{a}}
    + \psi' \exp\del{\frac{b^2}{4a}} \frac{b \sqrt\piup}{2 a^{3/2}}
    + \psi'' \exp\del{\frac{b^2}{4a}} \frac{[2a + b^2] \sqrt\piup}{8 a^{5/2}}
    \intertext{%
        We pull out the common terms to the front.
    }
    &= \sqrt\piup \exp\del{\frac{b^2}{4a}} \sbr{
        \psi \frac{1}{\sqrt a}
        + \psi' \frac{b }{2 a^{3/2}}
        + \psi'' \frac{[2a + b^2]}{8 a^{5/2}}
    }
\end{align*}

We can compute the inverse of $a$:
\[
    a = - \frac{\iup m + 2 \iup \epsilon q \alpha A'}{2 \epsilon \hbar}
    \iff
    \frac 1a
    = - \frac{2 \epsilon \hbar}{\iup m + 2 \iup \epsilon q \alpha A'}
    = \frac{2 \iup \epsilon \hbar}{m + 2 \epsilon q \alpha A'}.
\]
This does not have a singularity at $\epsilon = 0$ any more, which is
probably a good thing.

The $b^2$ is
\[
    b^2 = - \frac{q^2A^2}{\hbar^2}
\]
such that
\[
    \frac{b^2}{4a} = - \frac{\iup \epsilon q^2A^2}{2
    \hbar [m + 2 \epsilon q \alpha A']}.
\]

Since we are only interested in the first order of $\epsilon$, we can
expand the exponential function into a “$1 + x$” linear fashion. The
problem is that the $\sqrt{1/a}$ terms cannot be expanded because the first
derivative has a singularity at $\epsilon = 0$. So it could only be expanded in
terms of $\sqrt\epsilon$ which does not allow use to get an expansion in terms
of $\epsilon$. To give the form of equation~(6) from the problem set, we will
still have to get rid of the $\sqrt\pi$ that we got from the Gaussian
integration. This is where we do not know how to proceed.

To motivate equation~(6) a bit: Starting from the Schrödinger equation
\[
    \iup \hbar \pd{}t \psi(x, t) = \hat H \psi(x, t),
\]
we can derive the time evolution operator of a time-independent Hamiltonian
operator by solving the differential equation formally:
\[
    \psi(x, t) = \underbrace{\exp\del{-\frac{\iup}{\hbar} \hat H t}}_{U(t)}
    \psi(x, 0).
\]
Now we can expand this to first order around $t = 0$ and then relabel it with
$\epsilon$.
\[
    \psi(x, \epsilon) = \sbr{1 - \frac{\iup}{\hbar} \hat H \epsilon} \psi(x, 0).
\]
Rearranging the terms, we yield the version on the problem set:
\[
    \psi(x, \epsilon) = \psi(x, 0) - \iup \frac{\epsilon}{\hbar} \hat H \psi(x, 0).
\]

\end{document}

% vim: spell spelllang=en tw=79
